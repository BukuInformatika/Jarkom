%%%%%%%%%%%%%%
%% Run LaTeX on this file several times to get Table of Contents,
%% cross-references, and citations.

%% If you have font problems, you may edit the w-bookps.sty file
%% to customize the font names to match those on your system.

%% w-bksamp.tex. Current Version: Feb 16, 2012
%%%%%%%%%%%%%%%%%%%%%%%%%%%%%%%%%%%%%%%%%%%%%%%%%%%%%%%%%%%%%%%%
%
%  Sample file for
%  Wiley Book Style, Design No.: SD 001B, 7x10
%  Wiley Book Style, Design No.: SD 004B, 6x9
%
%
%  Prepared by Amy Hendrickson, TeXnology Inc.
%  http://www.texnology.com
%%%%%%%%%%%%%%%%%%%%%%%%%%%%%%%%%%%%%%%%%%%%%%%%%%%%%%%%%%%%%%%%

%%%%%%%%%%%%%
% 7x10
%\documentclass{wileySev}

% 6x9
\documentclass{wileySix}

\usepackage{graphicx}
\usepackage{listings}

\usepackage{color}
 
\definecolor{codegreen}{rgb}{0,0.6,0}
\definecolor{codegray}{rgb}{0.5,0.5,0.5}
\definecolor{codepurple}{rgb}{0.58,0,0.82}
\definecolor{backcolour}{rgb}{0.95,0.95,0.92}
 
\lstdefinestyle{mystyle}{
    backgroundcolor=\color{backcolour},   
    commentstyle=\color{codegreen},
    keywordstyle=\color{magenta},
    numberstyle=\tiny\color{codegray},
    stringstyle=\color{codepurple},
    basicstyle=\footnotesize,
    breakatwhitespace=false,         
    breaklines=true,                 
    captionpos=b,                    
    keepspaces=true,                 
    numbers=left,                    
    numbersep=5pt,                  
    showspaces=false,                
    showstringspaces=false,
    showtabs=false,                  
    tabsize=2,
    language=sh
}
 
\lstset{style=mystyle}

%%%%%%%
%% for times math: However, this package disables bold math (!)
%% \mathbf{x} will still work, but you will not have bold math
%% in section heads or chapter titles. If you don't use math
%% in those environments, mathptmx might be a good choice.

% \usepackage{mathptmx}

% For PostScript text
\usepackage{w-bookps}

%%%%%%%%%%%%%%%%%%%%%%%%%%%%%%%%%%%%%%%%%%%%%%%%%%%%%%%%%%%%%%%%
%% Other packages you might want to use:

% for chapter bibliography made with BibTeX
% \usepackage{chapterbib}

% for multiple indices
% \usepackage{multind}

% for answers to problems
% \usepackage{answers}

%%%%%%%%%%%%%%%%%%%%%%%%%%%%%%
%% Change options here if you want:
%%
%% How many levels of section head would you like numbered?
%% 0= no section numbers, 1= section, 2= subsection, 3= subsubsection
%%==>>
\setcounter{secnumdepth}{3}

%% How many levels of section head would you like to appear in the
%% Table of Contents?
%% 0= chapter titles, 1= section titles, 2= subsection titles, 
%% 3= subsubsection titles.
%%==>>
\setcounter{tocdepth}{2}

%% Cropmarks? good for final page makeup
%% \docropmarks

%%%%%%%%%%%%%%%%%%%%%%%%%%%%%%
%
% DRAFT
%
% Uncomment to get double spacing between lines, current date and time
% printed at bottom of page.
% \draft
% (If you want to keep tables from becoming double spaced also uncomment
% this):
% \renewcommand{\arraystretch}{0.6}
%%%%%%%%%%%%%%%%%%%%%%%%%%%%%%

%%%%%%% Demo of section head containing sample macro:
%% To get a macro to expand correctly in a section head, with upper and
%% lower case math, put the definition and set the box 
%% before \begin{document}, so that when it appears in the 
%% table of contents it will also work:

\newcommand{\VT}[1]{\ensuremath{{V_{T#1}}}}

%% use a box to expand the macro before we put it into the section head:

\newbox\sectsavebox
\setbox\sectsavebox=\hbox{\boldmath\VT{xyz}}

%%%%%%%%%%%%%%%%% End Demo


\begin{document}


\booktitle{Cerdas Menguasai Git}
\subtitle{Dalam 24 Jam}

\authors{Rolly M. Awangga\\
\affil{Informatics Research Center}
%Floyd J. Fowler, Jr.\\
%\affil{University of New Mexico}
}

\offprintinfo{Cerdas Menguasai Git, First Edition}{Rolly M. Awangga}

%% Can use \\ if title, and edition are too wide, ie,
%% \offprintinfo{Survey Methodology,\\ Second Edition}{Robert M. Groves}

%%%%%%%%%%%%%%%%%%%%%%%%%%%%%%
%% 
\halftitlepage

\titlepage


\begin{copyrightpage}{2019}
%Survey Methodology / Robert M. Groves . . . [et al.].
%\       p. cm.---(Wiley series in survey methodology)
%\    ``Wiley-Interscience."
%\    Includes bibliographical references and index.
%\    ISBN 0-471-48348-6 (pbk.)
%\    1. Surveys---Methodology.  2. Social 
%\  sciences---Research---Statistical methods.  I. Groves, Robert M.  II. %
%Series.\\
%
%HA31.2.S873 2007
%001.4'33---dc22                                             2004044064
\end{copyrightpage}

\dedication{`Jika Kamu tidak dapat menahan lelahnya belajar, 
Maka kamu harus sanggup menahan perihnya Kebodohan.'
~Imam Syafi'i~}

\begin{contributors}
\name{Rolly Maulana Awangga,} Informatics Research Center., Politeknik Pos Indonesia, Bandung,
Indonesia



\end{contributors}

\contentsinbrief
\tableofcontents
\listoffigures
\listoftables
\lstlistoflistings


\begin{foreword}
Sepatah kata dari Kaprodi, Kabag Kemahasiswaan dan Mahasiswa
\end{foreword}

\begin{preface}
Buku ini diciptakan bagi yang awam dengan git sekalipun.

\prefaceauthor{R. M. Awangga}
\where{Bandung, Jawa Barat\\
Februari, 2019}
\end{preface}


\begin{acknowledgments}
Terima kasih atas semua masukan dari para mahasiswa agar bisa membuat buku ini 
lebih baik dan lebih mudah dimengerti.

Terima kasih ini juga ditujukan khusus untuk team IRC yang 
telah fokus untuk belajar dan memahami bagaimana buku ini mendampingi proses 
Intership.
\authorinitials{R. M. A.}
\end{acknowledgments}

\begin{acronyms}
\acro{ACGIH}{American Conference of Governmental Industrial Hygienists}
\acro{AEC}{Atomic Energy Commission}
\acro{OSHA}{Occupational Health and Safety Commission}
\acro{SAMA}{Scientific Apparatus Makers Association}
\end{acronyms}

\begin{glossary}
\term{Hardware}Merupakan komponen dari sebuah komputer yang sifatnya bisa dapat dilihat secara kasat mata dan bisa diraba secara langsung dan hadware berfungsi untuk mendukung proses berjalanya komputer.

\term{Software}Merupakan suatu bagian dari sistem komputer yang tidak memiliki wujud fisik seperti hardware tetapi software merupakan sebuah nyawa komputer supaya dapat menjalankan perintah dari user.

\term{Internet}Merupakan jaringan komputer yang dimana satu jaringan dengan yang lain dapat saling terhubung untuk keperluan komunikasi dan informasi atau dapat disimpulkan internet dapat menghubungkan suatu media elektronik dengan media lainya.

\term{Server}Adalah sebuah sistem komputer yang menyediakan jenis layanan (service) tertentu dalam sebuah jaringan komputer server juga menjalankan perangkat lunak administratif yang mengontrol akses terhadap jaringan tersebut.

\term{client}Sistem atau proses yang dapat melakukan permintaan (request) data ke server.

\term{broadcast}Adalah sebuah pengiriman data dimana data akan dikirim langsung ke banyak titik sekaligus tanpa melakukan pengecekan,
Broadcast merupakan sebuah pengiriman data dimana data akan dikirim ke titik yang tidak sedikit secara bersamaan.

\term{switch}Sebuah perangkat jaringan pada komputer yang menghubungkan sebuah perangkat pada sebuah jaringan komputer dengan menggunakan pertukaran paket untuk menerima data, dan akan memproses untuk meneruskan data ke perangkat yang akan dituju.

\term{Hub}Adalah sebuah perangkat yang berfungsi untuk menghubungkan komputer yang satu dengan komputer lainnya dalam suatu sistem jaringan. 

\term{Bridge}Merupakan sebuah komponen jaringan yang banyak dipergunakan untuk memperluas jaringan atau membuat segmen jaringan.

\end{glossary}

\begin{symbols}
\term{A}Amplitude

\term{\hbox{\&}}Propositional logic symbol 

\term{a}Filter Coefficient

\bigskip

\term{\mathcal{B}}Number of Beats
\end{symbols}

\begin{introduction}

%% optional, but if you want to list author:

\introauthor{Rolly Maulana Awangga, S.T., M.T.}
{Informatics Research Center\\
Bandung, Jawa Barat, Indonesia}

Pada era disruptif  \index{disruptif}\index{disruptif!modern} 
saat ini. git merupakan sebuah kebutuhan dalam sebuah organisasi pengembangan perangkat lunak.
Buku ini diharapkan bisa menjadi penghantar para programmer, analis, IT Operation dan Project Manajer.
Dalam melakukan implementasi git pada diri dan organisasinya.

Rumusnya cuman sebagai contoh aja biar keren\cite{awangga2018sampeu}.

\begin{equation}
ABC {\cal DEF} \alpha\beta\Gamma\Delta\sum^{abc}_{def}
\end{equation}

\end{introduction}

%%%%%%%%%%%%%%%%%%Isi Buku_

\chapter{Judul Bagian Pertama}
\section{Pengenalan Jaringan Komputer}
 Jaringan Komputer merupakan kumpulan dari beberapa PC(Personal Computer) atau peripheral yang saling terhubung melalui media transmisi(melalui kabel atau nirkabel) dan melakukan akses bersama  terhadap suatu resource.
 \par Secara lebih sederhana, jaringan komputer dapat diartikan sebagai sekumpulan komputer berserta mekanisme dan prosedurnya yang saling terhubung dan berkomunikasi.  Komunikasi yang dilakukan oleh komputer tersebut dapat berupa transfer berbagai data, instruksi, dan informasi dari satu komputer ke komputer yang lain \cite{irawan2012analisis}.

 resource(sumber daya) tersebut terdiri dari:
 \begin{enumerate}
   \item Hardware, seperti: Printer,mesin fax, store device.
   \item Software, seperti: game, pemprograman client server, multi user, mail server
   \item Stored, Seperti: frontend atau backend
   \item Internet,Seperti: dial atau wireless
 \end{enumerate}

Keuntungan Jaringan Komputer:
\begin{enumerate}
  \item Lebih hemat dalam biaya pengadaan dan pemeliharaan
  \item Memungkinkan management sumber daya lebih efisien
  \item Mempertahankan kualitas Informasi agar tatap handal
  \item Memungkinkan Kelompok kerja berkomunikasi lebih efisien
  \item Keamanan data lebih terjamin
\end{enumerate}

Type Jaringan Komputer
 Pada dasarnya seseorang menentukan type jaringan komputer karena beberapa alasan, diantaranya adalah:
 \begin{enumerate}
   \item Disesuaikan dengan kebutuhan kita dalam membuat jaringan komputer.
   \item Tergantung kepada jumlah pengguna yang akan melakukan sharing data.
   \item keamanan (security) dari masing-masing jaringan.
   \item Mempertimbangkan dalam biaya pengadaan dari jaringan komputer
   \item Sumber daya admin menentukan jaringan komputer.
   \item Bentuk dari organisasi yang terbentuk.
 \end{enumerate}

Server Based
pada type jaringan komputer server based di perlukan satu atau lebih komputer khusus yang di sebut server untuk mengatur lalu lintas data atau informasi dalam jaringan komputer.komputer-komputer selain server dinamakan client. server yaitu komputer yang menyediakan fasilitas bagi komputer-komputer lain, sedangkan client yaitu komputer-komputer yang menerima atau menggunakan fasilitas yang di sediakan oleh server.

server dibedakan atas dua macam yaitu dedicated server(server bisa jadi client) dan undedicated server(server mutlak,tidak bisa jadi client).

macam-macam undedicated server:
\begin{itemize}
  \item DNS (Domain Name Service) yaitu server yang di gunakan untuk mengkonfersi penamaan IP address menjadi penanaman yang lebih familier (umum).
  \item DHCP (Dinamic Host Configurasi Protocol) yaitu server yang di gunakan untuk memberikan pengalaman IP address secara otomatis yang bersifat random. cara kerja random. Cara kerjanya pertama request (permintaan) kemudian dibroadcast.
  \item  FTP (file Transfer Protokol) yaitu server yang di gunakan untuk mengola jenis file/folder supaya data yang dinformasikan terpusat
  \item Mail Server merupakan jenis data dalam bentuk surat elektronik dibedakan menjadi dua yitu dalam bentuk text POP V3 (post office protocol) dan dalam bentuk web SMTP (simple Mail Transfer Protocol)
  \item Web server yaitu server yang di gunakan untuk mengelola data web yang bersifat dinamis.
  \item Database server yaitu server dalam bentuk file database.
\end{itemize}

ciri-ciri Server based
\begin{itemize}
  \item Operating System yang di gunakan berjenis network
  \item Perangkat yang di gunakan lebih dari 10 PC
  \item Terdapat komputer yang di jadikan sebagai pengontrol(server)\cite{wahyono2007building}
\end{itemize}

kelebihan Server based
\begin{itemize}
  \item terpusatnya penyedia resource
  \item Sharing data lebih efektif dan efesien
  \item System keamanan dan admistrasi jaringan lebih baik
\end{itemize}

Ciri-ciri Server Based
\begin{itemize}
  \item Operating System yang di gunakan berjenis network
  \item Perangkat yang di gunakan lebih dari 10 PC
  \item Terdapat Komputer yang di jadikan sebagai pengontrol(server)
\end{itemize}

Kelebihan Server Based
\begin{itemize}
  \item Terpusatnya penyedia resource
  \item Sharing data lebih efektif dan efesien
  \item System keamanan dan administrasi jaringan lebih baik
\end{itemize}

\subsection {Jenis-jenis Jaringan Komputer}
 Jenis jenis jaringan komputer dilihat berdasarkan ruang lingkup dan luas jangkuannya,di bedakan menjadi beberapa macam,yaitu:
\begin{itemize}
  \item Local Area Network(LAN)
   LAN adalah suatu system jaringan di mana setiao komputer atau perangkat keras dan perangkat lunak di gabungkan agar dapat saling berkomunikasi (terintegrasi) dalam area kerja tertentu dengan menggunakan data dan program yang sama,juga mempunyai kecepatan transfer data lebih cepat. Ruang Lingkup LAN anatr ruangan,gedung,kantor
\end{itemize}

\subsection {Topologi Jaringan Komputer}
    Topologi jaringan komputer adalah jaringan yang berhubungan dengan susuanan fisik semua jaringan komputer, baik server maupun client yang menentukan design,susunan,bentuk dari cara penempatan komputer(peripheral) kedalam jaringan-jaringan komputer.Topologi akan membentuk:
\begin{enumerate}
  \item Jenis alat yang di gunakan
  \item kemampuan dari peralatan
  \item Pertembuhan dari jaringan komputer
  \item Bagaimana jaringan tersebut diatur
\end{enumerate}

jenis alat-alat yang di gunakan,syaratnya:
\begin{itemize}
  \item Minimal 2 PC
  \item Adanya Operating System
  \item Adanya Network Interface Card(NIC)
  \item Driver NIC
  \item Media Transmisi
  \item Konsetrator(penghubung
  \item Access Point(tanpa kabel)
  \item Hub
  \item switch
  \item Repeater(Penguat signal)
  \item Router (Pembeda IP Address)
  \item Gatway(Perbedaan Arsitektur)
  \item Bridge (penghubung perbedaan topologi)
  \item Modern (modulasi de Modelator)
\end{itemize}

Topologi jaringan dibagi menjadi dua macam yaitu:
\begin{enumerate}
  \item Fisika
        Topologi ini menjelaskan tentang bentuk dari jaringan komputer yang dapat dilihat secara fisik/nyara.
\end{enumerate}

\begin{enumerate}
  \item Topologi Bus
        Masing-masing server dan workstastion di hubungkan pada sebuah kabel yang di sebut trunk atau backbone, kabel untuk menghubungkan jaringan ini biasanya menggunakan kabel Coaxial (kabel BNC). setiap server dan workstation yang di sambungkan pada bus menggunakan konektor T.
\end{enumerate}
 pada kedua ujung dari kabel harus diberi terminator berupa resistor yang memiliki resistansi khusus sebesar 50 Ohm yang berwujud sebuah konektor.Apabila resistansi kabel dibawah maupun di atas 50 Ohm,maka server tidak akan bisa bekerja secara maksimal dalam melayani jaringan,sehingga akses user dan client menjadi menurun.
 kelebihan jaringan topologi bus:
 \begin{itemize}
   \item Penggunaan kabel yang sedikit sehingga terlihat sederhana.
   \item Pengembangan jaringan mudah.
 \end{itemize}
 Kekurangan jaringan topologi
 \begin{itemize}
   \item Membutuhkan repeater untuk jarak jaringan yang terlalu jauh.
   \item jaringan akan terganggu apabila salah satu komputer mengalami kerusakan.
   \item Deteksi kesalahan sangat kecil sehingga apabila terjadi gangguan maka sulit sekali mencari kesalahan tersebut.
   \item Terjadi antrian data
 \end{itemize}

\begin{enumerate}
  \item Topologi Star
        Pada topologi in, setiap komputer(node) dalam jaringan terhubung ke sebuah pusat jaringan,yang biasa berupa hub,switch, dan juga berupa komputer. setiap workstasion dihubungkan ke server menggunakan suatu konsentrator. masing-masing workstastion tidak saling berhubungan.jadi setiap user yang terhubung ke server tidak akan dapat berinteraksi dan melakukan apa-apa sebelum server kita di hidupkan.apabila komputer server mati maka semua koneksi jaringan akan terputus.
\end{enumerate}
kelebihan jaringan topologi star:
\begin{itemize}
  \item Mudah dalam medeteksi kesalahan jaringan karena control jaringan terpusat.
  \item fleksibe dalam hal pemasangan jaringan baru tanpa mempengaruhi jaringan yang lain.
  \item apabila salah satu kabel koneksi user terpusat maka hanya user yang bersangkutan saja yang tidak akan berfungsi dan tidak mempengaruhi user yang lain.
\end{itemize}
kekurangan jaringan topologi star:
\begin{itemize}
  \item Boros dalam pemakaian kabel jika kita hubungkan dengan jaringan yang lebih besar dan luas.
  \item Control hanya terpusat pada hub/switch sehingga operasionalnya perlu ditangani secara khusus.
\end{itemize}

\begin{enumerate}
  \item topologi cincin atau yang sering disebut dengan ring topologi adalah topologi jaringan di mana setiap komputer yang terhubung membuat lingkaran. dengan artian setiap komputer yang terhubung kedalam satu jaringan saling terkoneksi ke dua komputer lainnya sehingga membentuk satu jaringan yang sama dengan bentuk cincin.
      Pada setiap komputer akan dihubungkan dan di jadikan repeater(penguat signal). komputer yang diberi frame berhak mengirim data dan komputer yang lain menjadi repeater. pada topologi ring terdapat token frame yang saling berkesinambungan dan pada prinsipnya menggunakan CSMA/CD(Carrier Sense Multyple Access/Collection Detection).
\end{enumerate}
Kelebihan jaringan topologi ring adalah:
\begin{itemize}
  \item Hemat kabel
  \item Dapat mengisolasi kesalahan dari suatu workstation kekurangan jaringan topolgi ring
  \item Sangat peka terhadap kesalahan jaringan walaupun sekecil apapun.
  \item Sukar untuk mengembangkan jaringan,sehingga jaringan tersebut tampak menjadi kaku.
  \item Biaya pemasangan Lebih besar.
\end{itemize}

\begin{enumerate}
  \item Logik
        sedanhkan topologi ini menjelaskan tentang bagaimana signal akan melewati komputer didalam jaringan. Arsitektur ini terus di kembangkan sampai saat ini.
\end{enumerate}

\begin{enumerate}
  \item  Token Ring
         Token ring memanfaatkan topologi ring. sebuah token bebas mengalir dalam jaringan. Apabila suatu node ingin mengirim paket data, maka paket data yang akan dikirim ditempel pada token, token kemudian akan membawa paket data tersebut pada tujuannya. pada waktu token terisi data, node lain tidak dapat menggunakan token tersebut sampai token menyelesaikan tugas mengirimkan paket data. apabila paket data telah di sampaikan pada tujuan,node pengguna tadi melepaskan token untuk dipakai oleh node lain. cara kerja dinamakan token passing scheme.
\end{enumerate}
ciri-ciri token ring:
\begin{itemize}
  \item Kecepatannya 1 Mbps, 4 Mbps hingga 16 Mbps.
  \item untuk menghubungkan station membutuhkan multistation Access Unit(MAU)
\end{itemize}

\begin{enumerate}
  \item Arsitektur ArcNet(Attached ResourceComputer Network)
        Didesain untuk system komputer Datapoint dan dikembangkan oleh Datapoint Corporation. Saat pertama didesain Arcnet menggunakan ukuran frame kecil 508 byte. Arcnet didesain agar handal dan tahan terhadap kerusakan pada kabel dan station.
\end{enumerate}
Ciri-ciri ArcNet adalah:
\begin{itemize}
  \item Topologi fisik yang di gunakan biasanya topologi Bus atau Star
  \item Prinsip kerjanya menggunakan token passing scheme atau broadcast
  \item Implementasinya menggunakan kabel coaxial RG-62
  \item Kecepatan mulai dari 2.5 Mbps hingga 20 Mbps.
\end{itemize}

\begin{enumerate}
  \item Merupakan implementasi metode CSMA/CD yang dikembangkan
        tahun 1960 pada proyek wire.sejak tahun 1978 IEEE (Institute Of Electrical and Electronics Engineers) telah melakukan
        standarisasi system Ethernet. Kecepatan Transmisi data saat ini antara 10 sampai 100 Mbps.
\end{enumerate}

\begin{enumerate}
  \item FDDI(Fiber Distribusi Data Interface)
        Merupakan suatu protocol jaringan yang menghubungkan antara
        dua atau beberapa jaringan yang jaraknya berdekatan ataupun berjauhan adapun metode yang di gunakan dalam FDDI adalah metode token ring.
\end{enumerate}

ciri-ciri FDDI adalah:
\begin{itemize}
  \item  Implementasinya menggunakan kabel fiber optic
  \item  Memiliki kecepatan 100 Mbps
  \item  Tidak compatibel dengan Ethernet tapi Ethernet dapat
         dienkapsulasi dalam paket FDDI
  \item  Bekerja berdasarkan dua ring concentris
  \item  Apabila salah satu ring atau node putus maka ring yang lain
         dapat berfungsi sebagai back up.
\end{itemize}

\begin{enumerate}
  \item ATM(Asynchronous Transfer MOde)
        Merupakan teknologi jaringan berkecepatan tinggi yang mampu
        mengirim data,suara dan video secara real time.ATM juga biasa di sebut Cell Relay. ATM merupakan interface transfer paket yang effesien. ATM menggunakan paket-paket dengan ukuran tertentu yang di sebut dengan cell. karena menggunakan ukuran tertentu ini, ATM menghasilkan skema yang efisien bagi pentransmisian pada jaringan berkecepatan tinggi. ATM menyediakan layanan real time dan non real time.
\end{enumerate}

\subsection{Topologi jaringan komputer}
Topologi merupakan suatu pola hubungan antara terminal dalam jaringan komputer.pola ini sangat erat kaitannya dengan metode access dan media pengiriman yang di gunakan. Topologi yang ada sangatlah tergantung dengan letak geografis dari masing-masing terminal,kualitas kontrol yang di butuhkan dalam komunikasi ataupun penyampaian pesan, serta kecepatan dari pengiriman data. dalam definisi topologi terbagi menjadi dua, yaitu Topologi logik (logical topology) yang menunjukan bagaimana suatu media di akses oleh host.

Adapun topologi fisik yang umum di gunakan dalam membangun sebuah jaringan adalah:
\begin{itemize}
  \item Point to point(Titik ke Titik)
        jaringan kerja titik ke titik merupakan jaringan kerja yang paling sederhana tetapi dapat di gunakan secara luas. begitu sederhananya jaringan ini, sehingga sering kali tidak dianggap sebagai suatu jaringan tetapi hanya merupakan komunikasi biasa.

        dalam hal ini, kedua simpul mempunyai kedudukan yang setingkat, sehingga simpul manapun dapat memulai dan mengendali hubungan dalam jaringan tersebut. data kirim dari satu simpul langsung kesimpul lainnya sebagai penerima,misalnya antara terminal dengan cpu.
  \item  star Network (jaringan bintang)
         dalam konfigurasi bintang, beberapa peralatan yang ada akan di hubungkan kedalam satu pusat komputer.kontrol yang ada akan di pusatkan pada sutu titik, seperti misalnya mengatur beban kerja serta pengaturan sumber daya yang ada. semua link harus berhubungan dengan pusat apabila ingin menyalurkan dara kesimpul lainnya yang di tuju. dalam hal ini, bila pusat mengalami gangguan, maka semua terminal juga akan terganggu. model jaringan bintang ini relative sangat sederhana, sehingga banyak di gunakan oleh pihak per-bank-kan yang biasanya mempunyai banyak kantor cabang yang tersebar di berbagai lokasi. dengan adanya konfigurasi bintang ini, maka segala macam kegiatan yang ada di kantor cabang dapatlah di kontrol dan di koordinasikan dengan baik. di samping itu, dunia pendidikan juga banyak memanfaatkan jaringan bintang ini guna mengontrol kegiatan anak didik mereka.
  \item Ring Networks (jaringan Cincin)
  \     pada jaringan ini terdapat beberapa peralatan saling
        di hubungkan satu dengan lainnya dan pada akhirnya akan membentuk bagan seperti halnya sebuah cincin. jaringan cincin tidak memiliki suatu titik yang bertindak sebagai pusat ataupun pengantur lalu lintas data, semua simpul mempunyai tingkatan yang sama. data yang di kirim akan berjalan melewati beberapa simpul sehingga sampai pada simpul yang di tuju. dalam menyampaikan data, jaringan bisa bergerak dalam satu ataupun dua arah.

        walaupun demikian, data yang ada tetap bergerak satu arah dalam satu saat. pertama, pesan ada akan di sampaikan dari titik ke titik lainnya dalam satu arah. apabila di temui kegagalan, misalnya terdapat kerusakan pada peralatan yang ada, maka data yang akan di kirim dengan cara kedua, yaitu pesan kemudian di transmisi dalam arah yang berlawanan, dan pada akhirnya bisa berakhir pada tempat yang di tuju.
  \item tree Network (jaringan pohon)
        pada jaringan pohon, terdapat beberapa tingkatan simpul (node).pusat atau simpul yang lebih tinggi tingkatanya, dapat mengatur simpul lain yang lebih rendah tingkatanya.
\end{itemize}

topologi logik pada umumnya terbagi menjadi dua tipe,yaitu:
\begin{enumerate}
  \item Topologi Broadcast
        secara sederhana dapat di gambarkan yaitu suatu host yang mengirimkan data kepada seluruh host lain pada media jaringan.
  \item Topologi token passing
        mengatur pengiriman data pada host melalui media dengan menggunakan token yang secara teratur berputar pada seluruh host. host hanya dapat mengirimkan data hanya jika host tersebut memiliki token. dengan token ini, collision dapat di cegah.
\end{enumerate}

\subsection{Bentuk komunikasi data}
\begin{itemize}
  \item Simpleks line (komunikasi satu arah)
        Merupakan bentuk saluran komunikasi yang paling murah, di mana komunikasi jenis ini hanya bisa berlangsung satu arah, dengan demikian pengirim informasi tidak bisa bertindak ataupun berubah menjadi penerima informasi, demikian pula sebaliknya.

        walaupun murah, jenis ini simpleks line jarang dipergunakan untuk komunikasi data,kalapun terpaksa hanya dipergunakan untuk hubungan antara CPU dengen printer, di mana printer hanya akan bertindak sebagai penerima informasi dari CPU. dalam kehidupan sehari-hari, kita bisa melihat radio panggil (pager) yang menggunakan transmisi-line dengan bentuk simpleks
  \item half-Dupleks(dua Arah bergantian).
        hal-dupleks line menginjika transmisi data dilakukan dalam dua arah, tetepi tidak dalam waktu yang bersamaan. jika line yang ada sedang mengirim data, misalnya dari terminal ke CPU, maka line yang bersangkutan pada saat itu tidak bisa di gunakan untuk mengirim data kembali dari CPU keterminal.

        dalam kehidupan sehari-hari, kita bisa melihat radio-CB yang di gunakan oleh para satpam ataupun anggota kepolisian. Radio-CB yang mereka pergunakan, menggunakan bentuk saluran half-dupleks sehingga pada saat pembicaraan berlangsung, sang pembicaraan harus menekan tombol tertentu agar suara yang di kirimkan bisa di salurkan kepada penerima. apabila di rasa cukup, maka pembicara akan mengucapkan kata "ganti"
        sebagai tanda bahwa saluran tersebut bisa di gunakan oleh lawan berbicaranya.
  \item Full-Dupleks (dua arah penuh)
        di dalam komunikasi ini, penerima dan pengirim informasi bisa secara serentak melakukan kegiatan bersama-sama, ataupun saling bertukar posisi dari penerima menjadi pengirim berita dan sebaliknya. data dalam hal ini dapat di kirim dari dua arah pada saat yang bersamaan
\end{itemize}

\subsection{Media Transmisi Data}
\begin{itemize}
  \item kabel Twisted Pair/uritan.
        Kabel jenis ini merupakan kabel yang paling luas penggunaanya karena di pergunakan untuk jaringan telpon.
        kabel ini terbuat dari tembaga di mana beberapa pasang kabel di-untir dan di jadikan satu. Guna mempertinggi kualitas kabel, seringkali setiap pasang kabel akan saling di-untir sehingga disebut sebagai untir-an.
  \item kabel koaksial
        pada jenis ini,kabel utama yang terbuat dari tembaga akan di keliling oleh anyaman halus kabel tembaga lainnya dan di antaranya terdapat isolasi. dari sudut harga, kabel ini lebih mahal apabila dibandingkan dengan kabel untiran,tetepi kualitas yang di berikan juga lebih baik.
  \item Fiber Optic(serat optik)
        dewasa ini terdapat usaha untuk menggunakan cahaya sebagai media komunikasi. data yang ada akan di bawa oleh cahaya, dan untuk menyalurkan cahaya yang membawa data tersebut, di perlukan adanya suatu jenis kabel yang khusus, dan kabel inilah yang di sebut sebagai fiber optic cable ataupun serat fiber. fiber optic terdiri atas suatu gelas fiber yang sangat tipis dan dapat di pergunakan untuk menyalurkan data dalam jumlah dan kecepatan yang sangat tinggi.
  \item Gelombang Radio-AM
        Sinyal yang berbentuk analog, juga dapat di transmisikan melalui udara, seperti misalnya: gelombang radio. AM-Radio
        yang merupakan singkatan dari Amplitude Modulation, dapat menangkap sinyal pada frekwensi yang sama, dan dengan kekuatan dan amplitude yang di miliknya, dapatlah menggerakan informasi ke arah yang di tuju.
  \item Pemancar Radio-FM/Station Televisi.
        Pemancar radio-FM dan station televisi juga dapat di gunakan untuk menyalurkan gelombang analog. dalam hal ini, Station televisi ataupun pemancar radio-FM (frekwensi Modulation)
        akan mendiami gelombang antara 54 hingga 806 megahertz(milions of cycles per second)
  \item Radio komunikasi Gelombang Pendek.
        dalam hal ini, radio komunikasi gelombang pendek banyak di gunakan oleh kalangan tertentu, misalnya ORARI ataupun kepolisian, juga dapat di manfaatkan untuk membawa sinyal analog ketempat yang di tiju. Radio komunikasi gelombang pendek memliki frekwensi yang lebih tinggi jika di banding
        dengan frekwensi yang di miliki oleh pemancar radio-AM
  \item telephone Celluar
        Telpon celuler ataupun genggam, ataupun telfon mobil yang bekerja pada frekwensi 825 hingga 890 megahertz, juga dapat di maanfaatkan sebagai suatu media transmsi komunikasi data.
  \item Gelombang Mikro
        Komunikasi data melalui gelombang elektro magnet (udara)
        yang paling banyak di gunakan adalah dengan gelombang mikro atau microwave. cara ini bisa menjangkau jarak yang sangat jauh, sehingga banyak kalangan industri ataupun pribadi yang menggunakannya untuk memindahkan/menyalurkan suara, video ataupun data komunikasi
  \item Satelit
        Pengguna satelit dirancang untuk mengurangi biaya pada pengiriman jarak yang sangat jauh. apabila di gunakan gelombang mikro, maka di perlukan banyak sekali station pemancar bumi yang harus di bangun. di samping itu juga
        harus diingat adanya lautan yang memisahkan daratan satu
        dengan lainnya. dengan menggunakan satelit, maka permasalahan yang ada bisa di atasi. Satelit secara umum
        bekerja pada frekwensi antara dua hingga 40 gigahertz
        (billion of hertz)
  \end{itemize}

\subsection{Protokol}
1. pengertian dasar Protokol
    protokol adalah sebuah aturan yang mendefinisikan beberapa fungsi yang ada dalam sebuah jaringan komputer, misalnya mengirim pesan,data, informasi dan fungsi lain yang harus di penuhi oleh sisi pengirim (transmitter) dan sisi penerima (receiver) agar komunikasi berlangsung dengan benar. selain itu protokol juga berfungsi untuk memungkinkan dua atau lebih komputer dapat berkomunikasi dengan bahasa yang sama.

hal-hal yang harus di perhatikan dalam protokol adalah sebagai berikut:
\begin{itemize}
  \item Syntax
        merupakan format data dan cara pengkodean yang di gunakan untuk mengkodekan sinyal.
  \item semantix
        Di gunakan untuk mengetahui maksud dari informasi yang di kirim dan mengkoreksi kesalahan yang terjadi dari informasi tadi.
  \item Timing
        di gunakan untuk mengetahui kecepatan transmisi data.
\end{itemize}

\subsection{Fungsi protokol}
 2.fungsi-fungsi protokol secara detail dapat di jelaskan sebagai berikut:
 \begin{itemize}
   \item Fragmentasi dan Reassembly
         fungsi dari fragmentasi dan reassembly adalah membagi informasi yang di kirim menjadi beberapa paket data pada saat isi pengirim mengirimkan informasi tadi dan setelah di terima maka sisi penerima akan menggabungkan lagi menjadi paket berita yang lengkap.
   \item Encaptulation
         fungsi dan ecaptulation adalah melengkapi berita yang di kirimkan dengan address, kode-kode koreksi dan lain-lain.
   \item Connection Control
         fungsi dari connection control adalah membangun hubungan komunikasi dari transmitter dan receiver, di mana dalam membangun hubungan ini termasuk dalam hal pengiriman data dan mengakhiri hubungan.
   \item Flow control
         fungsi dari flow control adalah mengatur perjalanan data transmitter ke receiver
   \item Error Control
         dalam pengiriman data tak lepas dari kesalahan,baik itu dalam proses pengiriman maupun pada waktu data itu di terima.fungsi dari error control adalah mengontrol terjadinya kesalahan yang terjadi pada waktu data dikirimkan.
   \item Transmission Service
         fungsi dari transmission service adalah memberi pelayanan komunikasi data khususnya yang berkaitan dengan prioritas dan keamanan serta perlindungan data.
 \end{itemize}

\subsection{TCP/IP}
3.TCP/IP bukanlah sebuah protokol tunggal tetapi satu kesehatan protokol dan utility.setiap protokol dalam kesatuan ini memiliki aturan yang spesifik. protokol ini dikembangkan oleh ARPA(Advanced Research Project Agency) untuk departement pertahanan Amerika serikat pada tahun 1969.
ARPA menginginkan sebuah protokol yang memiliki karakter sebagai berikut:
\begin{itemize}
  \item Mampu menghubungkan berbagai jenis sistem operasi.
  \item Dapat diandalkan dan mampu mendukung komunikasi kecepatan tinggi
  \item Routable dan scalable untuk memenuhi jaringan yang kompleks dan luas.Sebuah alamat TCP/IP adalah nilai biner berukuran 32 Bit yang di berikan kesetiap host dalam sebuah jaringan. Nilai ini di gunakan untuk mengenali jaringan di mana host yang terhubung jadi satu pada sebuah internet work harus memiliki satu alamat unik TCP/IP.
\end{itemize}

Setiap alamat terbagi atas dua komponen:
\begin{itemize}
  \item Network ID
        ini adalah bagian dari alamat IP yang mewakili jaringan fisik dari host (nama jalan dari rumah). Setiap komputer dalam segmen jaringan tertentu akan memiliki ID jaringan yang sama.
  \item Node ID
        ini adalah bagian yang mewakili bagian individu dari alamat(nomor rumah). Bila komputer disegment jaringan memiliki alamat, maka jaringan tersebut perlu miliki siapakah suatu paket itu.
        seperti yang di sebutkan di atas tadi bahwa nilai IP adalah nilai biner 32 bit. Nilai tersebut terbagi menjadi empat bagian 8 bit yang di sebut oktet. contoh alamat IP:202.149.240.66 dengan mengunakan contoh di atas, katakanlah administrator mensetup jaringan dengan semua komputer memiliki bagian nilai yang sama 202.149.240.xxx.konidisi ini yang di sebut network ID. Nomor pada XXX adalah node ID-nya.
        setiap alamat TCP/IP jatuh pada satu kelas alamat. kelas mewakili sebuah grup alamat yang segera dapat di kenal komponen software sebagai bagian dari sebuah jaringan fisik.
        misalkan, ambil contoh alamat TCP/IP berikut dan nilai binernya. 10.149.240.660001010.10010101.1111000.10000010 dengan memperhatikan tiga nilai biner yang pertama, bisa di katakan bahwa alamat ini termasuk class A.
\end{itemize}

\subsection{Konfigurasi Firewall}[IPtables]
Dasar Teori
firewall adalah sistem atau sekelompok sistem yang menetapkan kebijakan kendali akses antara dua jaringan. Secara prinsip, firewall dapat di anggap sebagai sepasang mekanisme : yang pertama memblok lalu lintas, yang kedua menginjikan lalu lintas jaringan, firewall dapat di gunakan untuk melindungi jaringan anda dari serangan jaringan oleh pihak luar, namun firewall tidak dapat melindungi dari serangan yang tidak melalui firewall dan serangan yang berada di dalam jaringan anda,serta firewall tidak dapat melindungi anda dari program-program aplikasi yang tertulis dengan buruk.

secara umum, firewall biasanya menjalankan fungsi:
\begin{itemize}
  \item Analisis dan filter paket
        Data yang berkomunikasi lewat protokol di internet, dibagi atas paket-paket. Firewall dapat menganalisis paket ini, kemudian mempermalukannya sesuai kondisi tertentu. misalnya, jika ada paket a maka akan di lakukan b. untuk filter paket, dapat di lakukan di linux tanpa program tambahan.
  \item Bloking isi dan protokol
        firewall dapat melakukan bloking terhadap isi paket, misalnyanya berisi applet jave,ActiveX, VBScript, Cookie.
  \item Autentika Koneksi dan enkripsi
        Firewall umumnya memiliki kemampuan untuk menjalankan enksipsi dalam autentikasi identitas user, integritas dari satu session, dan melapisi transfer data dari intipan pihak lain. enkripsi yang di makasud termasuk DES, tripple DES,SSl,IPSEC,SHA,MD5,BlowFish,IDEA dan sebagainya.
\end{itemize}

secara konseptual,terdapat dua macam firewall yaitu:
\begin{itemize}
  \item Network Level
        Firewall network level mendasarkan keputusan mereka pada alamat sumber,alamat tujuan dan prot yang terdapat dalam setiap paket IP. Network level firewall sangat cepat dan sangat transparan bagi pemakai. Application level firewall biasanya adalah host yang berjalan sebagai proxy server, yang tidak mengijinkan lalu lintas antara jaringan, dan melakukan logging dan auditing lalu lintas yang melaluinya
  \item Application level
        Application level firewall menyediakan laporan audit yang lebih rinci dan cenderung lebih memaksakan model keamanan yang lebih konservatif daripada network level firewall. Firewall ini bisa di katakan sebagai jembatan. Application- Proxy firewall biasanya berupa program khusus, misalnya squid
\end{itemize}

Konfigurasi Firewall[TCP Wrapper]
pada sistem operasi pada umumnya mempunyai program yang berjalan pada sistem operasi tersebut. pada sistem operasi linux,di kenal istilah service untuk mengantikan nama aplikasi secara global. Ada beberapa service yang di manage di linux:
\begin{itemize}
  \item System V scripts
        merupakan metode yang paling umum digunakan untuk menaging service, biasanya membutuhkan file konfigurasi. service distart dengan scrpit di/etc/init.d/. misalnya untuk network: /etc/init.d/network restart atau service network restart
  \item xinetd
        hanya beberapa service yang ada pada xinetd, service ini tidak memerlukan start/stop terhadap service. dan file konfigurasi yang biasa di pakai adalah:
        /etc/xinetd.conf -> Top level configuration file
        /etc/xinetd.d/service -> service specifition configuration
\end{itemize}

Network Scanning dan Probing

Dasar teori
 Server tugasnya adalah melayani client dengan menyediakan service yang di butuhkan. Server menyadiakan service dengan bermacam-macam kemampuan, baik untuk lokal maupun remote. server listening pada suatu port dan menunggu incomming connection ke port. koneksi bisa berupa lokal maupun remote.
    port sebernarnya suatu alamat pada stack jaringan kernel, sebagai cara di mana transport layer mengola koneksi dan melakukan pertukaran data antar komputer. port yang terbuka mempunyai resiko terkait dengan exploit. perlu dikelola port mana yang perlu di buka ini terbuka.
    port Scanner merupakan program yang didesain untuk menemukan layanan (service) apa saja dijalankan pada host jaringan. untuk mendapatkan akses ke host,cracker harus mengetahui titik-titik kelemahan yang ada. sebagai contoh, apabila cracker sudah mengetahui bahwa host menjalankan proses ftp server untuk mendapatkan akses. dari bagian ini kita dapat mengambil kesimpulan bahwa layanan yang tidak benar-benar diperlukan sebaiknya dihilangkan untuk memperkecil resiko keamanan yang memungkin terjadi.

 Type Scanning

 \begin{itemize}
   \item Connect scan (-sT)
         Jenis scan ini konek ke port sasaran dan menyelesaikan three-way handshake (SYN,SYN/ACK). Scan jenis mudah terdeteksi oleh sistem sasaran.
   \item -sS (TCP SYN scan)
          Paling populer dan merupakan scan default nmap. SYN scan juga sukar terdeteksi, karena tidak menggunakan 3 way handshake secara lengkap, yang di sebut sebagai teknik half open scanning. SYN scan juga efektif karena dapat membedakan 3 state port, yaitu open, filterd ataupun close. teknik ini di kenal sebagai half-openning scanning karena suatu koneksi TCP tidak sampai terbentuk. sebaliknya suatu paket SYN di kirimkan ke port sasaran. Bila SYN/ACK diterima dari port sasaran, kita dapat mengambil kesimpulan bahwa port itu berbeda dalam suatu LISTENING. Suatu RST/ACT akan dikirim oleh mesin yang melakukan scanning sehingga koneksi penuh tidak akan terbentuk. Teknik ini bersifat siluman di bandingkan TCP connect penuh, dan tidak aka tercatat pada log sistem sasaran.
   \item TCP FIN scan (-sF)
         Teknik ini mengirim suatu paket FIN ke port sasaran. berdasarkan RFC 793, sistem sasaran akan mengirim balik suatu RST untuk setiap port yang tertutup. Teknik ini hanya dapat dipakai pada stack TCP/IP berbabasis UNIX.
   \item TCP Xmas Tree Scam(-sX)
         Teknik ini mengirimkan suatu paket FIN,URG, dan PUSH ke port sasaran. Berdasarkan RFC 793, sistem sasaran akan mengembalikan suatu RST untuk semua port yang tertutup.
   \item TCP Null scan (-sN)
         Teknik ini membuat off semua flag. Berdasarkan RFC 793, Sistem sasaran akan mengirim balik suatu RST untuk semua port yang tertutup.
   \item TCP ACK scan(-sA)
         Teknik ini di gunakan untuk memetakan set aturan firewall. Dapat membantu menentukan apakah firewall itu merupakan suatu simple packet filter yang membolehkan hanya koneksi-koneksi tertentu (koneksi dengan bit set ACK) atau suatu firewall yang menjalankan advance packet filtering.
   \item TCP windows scan
         Teknik ini dapat mendeteksi port-port terbuka maupun terfilter/tidak terfilter pada sistem-sistem tertentu (sebagai contoh, AIX dan freeBSD) sehubungkan dengan anomali dari ukuran windows TCP yang di laporkan.
   \item TCP RPC scan
         teknik ini spesifik hanya pada system UNIX dan digunakan untuk mendeteksi dan mengindentifikasi port RPC (Remote Procedure Call) dan program serta normor versi yang berhubungan dengannya.
   \item UDP scan (-sU)
         Teknik ini mengirimkan suatu paket UDP ke port sasaran. Bila port sasaran memberikan respon berupa pesan (ICMP port uncreachable) artinya port ini tertutup. sebaliknya bila tidak menerima pesan di atas, kita dapat menyimpulkan bahwa port itu terbuka. karena UDP di kenal sebagai connectionless protocol, akurasi teknik ini sangat bergantung pada banyak hal sehubungan dengan penggunaan jaringan dan system resource. sebagai tambahan, UDP scanning merupakan proses yang amat lambat apabila anda mencoba men-scan suatu perangkat yang menjalankan packet filtering berbebban tinggi.
 \end{itemize} 
 
 

\chapter{Judul Bagian Kedua}
\section{Perintah Navigasi}
Perintah navigasi direktori


\chapter{Judul Bagian Ketiga}
\section{Perintah Navigasi}
Perintah navigasi direktori


\chapter{Judul Bagian Keempat}
\section{Perintah Navigasi}
Perintah navigasi direktori


\chapter{Judul Bagian Kelima}
\section{Perintah Navigasi}
Perintah navigasi direktori




\bibliographystyle{IEEEtran} 
%\def\bibfont{\normalsize}
\bibliography{references}


%%%%%%%%%%%%%%%
%%  The default LaTeX Index
%%  Don't need to add any commands before \begin{document}
\printindex

%%%% Making an index
%% 
%% 1. Make index entries, don't leave any spaces so that they
%% will be sorted correctly.
%% 
%% \index{term}
%% \index{term!subterm}
%% \index{term!subterm!subsubterm}
%% 
%% 2. Run LaTeX several times to produce <filename>.idx
%% 
%% 3. On command line, type  makeindx <filename> which
%% will produce <filename>.ind 
%% 
%% 4. Type \printindex to make the index appear in your book.
%% 
%% 5. If you would like to edit <filename>.ind 
%% you may do so. See docs.pdf for more information.
%% 
%%%%%%%%%%%%%%%%%%%%%%%%%%%%%%

%%%%%%%%%%%%%% Making Multiple Indices %%%%%%%%%%%%%%%%
%% 1. 
%% \usepackage{multind}
%% \makeindex{book}
%% \makeindex{authors}
%% \begin{document}
%% 
%% 2.
%% % add index terms to your book, ie,
%% \index{book}{A term to go to the topic index}
%% \index{authors}{Put this author in the author index}
%% 
%% \index{book}{Cows}
%% \index{book}{Cows!Jersey}
%% \index{book}{Cows!Jersey!Brown}
%% 
%% \index{author}{Douglas Adams}
%% \index{author}{Boethius}
%% \index{author}{Mark Twain}
%% 
%% 3. On command line type 
%% makeindex topic 
%% makeindex authors
%% 
%% 4.
%% this is a Wiley command to make the indices print:
%% \multiprintindex{book}{Topic index}
%% \multiprintindex{authors}{Author index}

\end{document}


