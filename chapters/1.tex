\section{Pengenalan Jaringan Komputer}
 Jaringan Komputer merupakan kumpulan dari beberapa PC(Personal Computer) atau peripheral yang saling terhubung melalui media transmisi(melalui kabel atau nirkabel) dan melakukan akses bersama  terhadap suatu resource.
 \par Secara lebih sederhana, jaringan komputer dapat diartikan sebagai sekumpulan komputer berserta mekanisme dan prosedurnya yang saling terhubung dan berkomunikasi.  Komunikasi yang dilakukan oleh komputer tersebut dapat berupa transfer berbagai data, instruksi, dan informasi dari satu komputer ke komputer yang lain \cite{irawan2012analisis}.

 resource(sumber daya) tersebut terdiri dari:
 \begin{enumerate}
   \item Hardware, seperti: Printer,mesin fax, store device.
   \item Software, seperti: game, pemprograman client server, multi user, mail server
   \item Stored, Seperti: frontend atau backend
   \item Internet,Seperti: dial atau wireless
 \end{enumerate}

Keuntungan Jaringan Komputer:
\begin{enumerate}
  \item Lebih hemat dalam biaya pengadaan dan pemeliharaan
  \item Memungkinkan management sumber daya lebih efisien
  \item Mempertahankan kualitas Informasi agar tatap handal
  \item Memungkinkan Kelompok kerja berkomunikasi lebih efisien
  \item Keamanan data lebih terjamin
\end{enumerate}

Type Jaringan Komputer
 Pada dasarnya seseorang menentukan type jaringan komputer karena beberapa alasan, diantaranya adalah:
 \begin{enumerate}
   \item Disesuaikan dengan kebutuhan kita dalam membuat jaringan komputer.
   \item Tergantung kepada jumlah pengguna yang akan melakukan sharing data.
   \item keamanan (security) dari masing-masing jaringan.
   \item Mempertimbangkan dalam biaya pengadaan dari jaringan komputer
   \item Sumber daya admin menentukan jaringan komputer.
   \item Bentuk dari organisasi yang terbentuk.
 \end{enumerate}

Server Based
pada type jaringan komputer server based di perlukan satu atau lebih komputer khusus yang di sebut server untuk mengatur lalu lintas data atau informasi dalam jaringan komputer.komputer-komputer selain server dinamakan client. server yaitu komputer yang menyediakan fasilitas bagi komputer-komputer lain, sedangkan client yaitu komputer-komputer yang menerima atau menggunakan fasilitas yang di sediakan oleh server.

server dibedakan atas dua macam yaitu dedicated server(server bisa jadi client) dan undedicated server(server mutlak,tidak bisa jadi client).

macam-macam undedicated server:
\begin{itemize}
  \item DNS (Domain Name Service) yaitu server yang di gunakan untuk mengkonfersi penamaan IP address menjadi penanaman yang lebih familier (umum).
  \item DHCP (Dinamic Host Configurasi Protocol) yaitu server yang di gunakan untuk memberikan pengalaman IP address secara otomatis yang bersifat random. cara kerja random. Cara kerjanya pertama request (permintaan) kemudian dibroadcast.
  \item  FTP (file Transfer Protokol) yaitu server yang di gunakan untuk mengola jenis file/folder supaya data yang dinformasikan terpusat
  \item Mail Server merupakan jenis data dalam bentuk surat elektronik dibedakan menjadi dua yitu dalam bentuk text POP V3 (post office protocol) dan dalam bentuk web SMTP (simple Mail Transfer Protocol)
  \item Web server yaitu server yang di gunakan untuk mengelola data web yang bersifat dinamis.
  \item Database server yaitu server dalam bentuk file database.
\end{itemize}

ciri-ciri Server based
\begin{itemize}
  \item Operating System yang di gunakan berjenis network
  \item Perangkat yang di gunakan lebih dari 10 PC
  \item Terdapat komputer yang di jadikan sebagai pengontrol(server)\cite{wahyono2007building}
\end{itemize}

kelebihan Server based
\begin{itemize}
  \item terpusatnya penyedia resource
  \item Sharing data lebih efektif dan efesien
  \item System keamanan dan admistrasi jaringan lebih baik
\end{itemize}

Ciri-ciri Server Based
\begin{itemize}
  \item Operating System yang di gunakan berjenis network
  \item Perangkat yang di gunakan lebih dari 10 PC
  \item Terdapat Komputer yang di jadikan sebagai pengontrol(server)
\end{itemize}

Kelebihan Server Based
\begin{itemize}
  \item Terpusatnya penyedia resource
  \item Sharing data lebih efektif dan efesien
  \item System keamanan dan administrasi jaringan lebih baik
\end{itemize}

\subsection {Jenis-jenis Jaringan Komputer}
 Jenis jenis jaringan komputer dilihat berdasarkan ruang lingkup dan luas jangkuannya,di bedakan menjadi beberapa macam,yaitu:
\begin{itemize}
  \item Local Area Network(LAN)
   LAN adalah suatu system jaringan di mana setiao komputer atau perangkat keras dan perangkat lunak di gabungkan agar dapat saling berkomunikasi (terintegrasi) dalam area kerja tertentu dengan menggunakan data dan program yang sama,juga mempunyai kecepatan transfer data lebih cepat. Ruang Lingkup LAN anatr ruangan,gedung,kantor
\end{itemize}

\subsection {Topologi Jaringan Komputer}
    Topologi jaringan komputer adalah jaringan yang berhubungan dengan susuanan fisik semua jaringan komputer, baik server maupun client yang menentukan design,susunan,bentuk dari cara penempatan komputer(peripheral) kedalam jaringan-jaringan komputer.Topologi akan membentuk:
\begin{enumerate}
  \item Jenis alat yang di gunakan
  \item kemampuan dari peralatan
  \item Pertembuhan dari jaringan komputer
  \item Bagaimana jaringan tersebut diatur
\end{enumerate}

jenis alat-alat yang di gunakan,syaratnya:
\begin{itemize}
  \item Minimal 2 PC
  \item Adanya Operating System
  \item Adanya Network Interface Card(NIC)
  \item Driver NIC
  \item Media Transmisi
  \item Konsetrator(penghubung
  \item Access Point(tanpa kabel)
  \item Hub
  \item switch
  \item Repeater(Penguat signal)
  \item Router (Pembeda IP Address)
  \item Gatway(Perbedaan Arsitektur)
  \item Bridge (penghubung perbedaan topologi)
  \item Modern (modulasi de Modelator)
\end{itemize}

Topologi jaringan dibagi menjadi dua macam yaitu:
\begin{enumerate}
  \item Fisika
        Topologi ini menjelaskan tentang bentuk dari jaringan komputer yang dapat dilihat secara fisik/nyara.
\end{enumerate}

\begin{enumerate}
  \item Topologi Bus
        Masing-masing server dan workstastion di hubungkan pada sebuah kabel yang di sebut trunk atau backbone, kabel untuk menghubungkan jaringan ini biasanya menggunakan kabel Coaxial (kabel BNC). setiap server dan workstation yang di sambungkan pada bus menggunakan konektor T.
\end{enumerate}
 pada kedua ujung dari kabel harus diberi terminator berupa resistor yang memiliki resistansi khusus sebesar 50 Ohm yang berwujud sebuah konektor.Apabila resistansi kabel dibawah maupun di atas 50 Ohm,maka server tidak akan bisa bekerja secara maksimal dalam melayani jaringan,sehingga akses user dan client menjadi menurun.
 kelebihan jaringan topologi bus:
 \begin{itemize}
   \item Penggunaan kabel yang sedikit sehingga terlihat sederhana.
   \item Pengembangan jaringan mudah.
 \end{itemize}
 Kekurangan jaringan topologi
 \begin{itemize}
   \item Membutuhkan repeater untuk jarak jaringan yang terlalu jauh.
   \item jaringan akan terganggu apabila salah satu komputer mengalami kerusakan.
   \item Deteksi kesalahan sangat kecil sehingga apabila terjadi gangguan maka sulit sekali mencari kesalahan tersebut.
   \item Terjadi antrian data
 \end{itemize}

\begin{enumerate}
  \item Topologi Star
        Pada topologi in, setiap komputer(node) dalam jaringan terhubung ke sebuah pusat jaringan,yang biasa berupa hub,switch, dan juga berupa komputer. setiap workstasion dihubungkan ke server menggunakan suatu konsentrator. masing-masing workstastion tidak saling berhubungan.jadi setiap user yang terhubung ke server tidak akan dapat berinteraksi dan melakukan apa-apa sebelum server kita di hidupkan.apabila komputer server mati maka semua koneksi jaringan akan terputus.
\end{enumerate}
kelebihan jaringan topologi star:
\begin{itemize}
  \item Mudah dalam medeteksi kesalahan jaringan karena control jaringan terpusat.
  \item fleksibe dalam hal pemasangan jaringan baru tanpa mempengaruhi jaringan yang lain.
  \item apabila salah satu kabel koneksi user terpusat maka hanya user yang bersangkutan saja yang tidak akan berfungsi dan tidak mempengaruhi user yang lain.
\end{itemize}
kekurangan jaringan topologi star:
\begin{itemize}
  \item Boros dalam pemakaian kabel jika kita hubungkan dengan jaringan yang lebih besar dan luas.
  \item Control hanya terpusat pada hub/switch sehingga operasionalnya perlu ditangani secara khusus.
\end{itemize}

\begin{enumerate}
  \item topologi cincin atau yang sering disebut dengan ring topologi adalah topologi jaringan di mana setiap komputer yang terhubung membuat lingkaran. dengan artian setiap komputer yang terhubung kedalam satu jaringan saling terkoneksi ke dua komputer lainnya sehingga membentuk satu jaringan yang sama dengan bentuk cincin.
      Pada setiap komputer akan dihubungkan dan di jadikan repeater(penguat signal). komputer yang diberi frame berhak mengirim data dan komputer yang lain menjadi repeater. pada topologi ring terdapat token frame yang saling berkesinambungan dan pada prinsipnya menggunakan CSMA/CD(Carrier Sense Multyple Access/Collection Detection).
\end{enumerate}
Kelebihan jaringan topologi ring adalah:
\begin{itemize}
  \item Hemat kabel
  \item Dapat mengisolasi kesalahan dari suatu workstation kekurangan jaringan topolgi ring
  \item Sangat peka terhadap kesalahan jaringan walaupun sekecil apapun.
  \item Sukar untuk mengembangkan jaringan,sehingga jaringan tersebut tampak menjadi kaku.
  \item Biaya pemasangan Lebih besar.
\end{itemize}

\begin{enumerate}
  \item Logik
        sedanhkan topologi ini menjelaskan tentang bagaimana signal akan melewati komputer didalam jaringan. Arsitektur ini terus di kembangkan sampai saat ini.
\end{enumerate}

\begin{enumerate}
  \item  Token Ring
         Token ring memanfaatkan topologi ring. sebuah token bebas mengalir dalam jaringan. Apabila suatu node ingin mengirim paket data, maka paket data yang akan dikirim ditempel pada token, token kemudian akan membawa paket data tersebut pada tujuannya. pada waktu token terisi data, node lain tidak dapat menggunakan token tersebut sampai token menyelesaikan tugas mengirimkan paket data. apabila paket data telah di sampaikan pada tujuan,node pengguna tadi melepaskan token untuk dipakai oleh node lain. cara kerja dinamakan token passing scheme.
\end{enumerate}
ciri-ciri token ring:
\begin{itemize}
  \item Kecepatannya 1 Mbps, 4 Mbps hingga 16 Mbps.
  \item untuk menghubungkan station membutuhkan multistation Access Unit(MAU)
\end{itemize}

\begin{enumerate}
  \item Arsitektur ArcNet(Attached ResourceComputer Network)
        Didesain untuk system komputer Datapoint dan dikembangkan oleh Datapoint Corporation. Saat pertama didesain Arcnet menggunakan ukuran frame kecil 508 byte. Arcnet didesain agar handal dan tahan terhadap kerusakan pada kabel dan station.
\end{enumerate}
Ciri-ciri ArcNet adalah:
\begin{itemize}
  \item Topologi fisik yang di gunakan biasanya topologi Bus atau Star
  \item Prinsip kerjanya menggunakan token passing scheme atau broadcast
  \item Implementasinya menggunakan kabel coaxial RG-62
  \item Kecepatan mulai dari 2.5 Mbps hingga 20 Mbps.
\end{itemize}

\begin{enumerate}
  \item Merupakan implementasi metode CSMA/CD yang dikembangkan
        tahun 1960 pada proyek wire.sejak tahun 1978 IEEE (Institute Of Electrical and Electronics Engineers) telah melakukan
        standarisasi system Ethernet. Kecepatan Transmisi data saat ini antara 10 sampai 100 Mbps.
\end{enumerate}

\begin{enumerate}
  \item FDDI(Fiber Distribusi Data Interface)
        Merupakan suatu protocol jaringan yang menghubungkan antara
        dua atau beberapa jaringan yang jaraknya berdekatan ataupun berjauhan adapun metode yang di gunakan dalam FDDI adalah metode token ring.
\end{enumerate}

ciri-ciri FDDI adalah:
\begin{itemize}
  \item  Implementasinya menggunakan kabel fiber optic
  \item  Memiliki kecepatan 100 Mbps
  \item  Tidak compatibel dengan Ethernet tapi Ethernet dapat
         dienkapsulasi dalam paket FDDI
  \item  Bekerja berdasarkan dua ring concentris
  \item  Apabila salah satu ring atau node putus maka ring yang lain
         dapat berfungsi sebagai back up.
\end{itemize}

\begin{enumerate}
  \item ATM(Asynchronous Transfer MOde)
        Merupakan teknologi jaringan berkecepatan tinggi yang mampu
        mengirim data,suara dan video secara real time.ATM juga biasa di sebut Cell Relay. ATM merupakan interface transfer paket yang effesien. ATM menggunakan paket-paket dengan ukuran tertentu yang di sebut dengan cell. karena menggunakan ukuran tertentu ini, ATM menghasilkan skema yang efisien bagi pentransmisian pada jaringan berkecepatan tinggi. ATM menyediakan layanan real time dan non real time.
\end{enumerate}

\subsection{Topologi jaringan komputer}
Topologi merupakan suatu pola hubungan antara terminal dalam jaringan komputer.pola ini sangat erat kaitannya dengan metode access dan media pengiriman yang di gunakan. Topologi yang ada sangatlah tergantung dengan letak geografis dari masing-masing terminal,kualitas kontrol yang di butuhkan dalam komunikasi ataupun penyampaian pesan, serta kecepatan dari pengiriman data. dalam definisi topologi terbagi menjadi dua, yaitu Topologi logik (logical topology) yang menunjukan bagaimana suatu media di akses oleh host.

Adapun topologi fisik yang umum di gunakan dalam membangun sebuah jaringan adalah:
\begin{itemize}
  \item Point to point(Titik ke Titik)
        jaringan kerja titik ke titik merupakan jaringan kerja yang paling sederhana tetapi dapat di gunakan secara luas. begitu sederhananya jaringan ini, sehingga sering kali tidak dianggap sebagai suatu jaringan tetapi hanya merupakan komunikasi biasa.

        dalam hal ini, kedua simpul mempunyai kedudukan yang setingkat, sehingga simpul manapun dapat memulai dan mengendali hubungan dalam jaringan tersebut. data kirim dari satu simpul langsung kesimpul lainnya sebagai penerima,misalnya antara terminal dengan cpu.
  \item  star Network (jaringan bintang)
         dalam konfigurasi bintang, beberapa peralatan yang ada akan di hubungkan kedalam satu pusat komputer.kontrol yang ada akan di pusatkan pada sutu titik, seperti misalnya mengatur beban kerja serta pengaturan sumber daya yang ada. semua link harus berhubungan dengan pusat apabila ingin menyalurkan dara kesimpul lainnya yang di tuju. dalam hal ini, bila pusat mengalami gangguan, maka semua terminal juga akan terganggu. model jaringan bintang ini relative sangat sederhana, sehingga banyak di gunakan oleh pihak per-bank-kan yang biasanya mempunyai banyak kantor cabang yang tersebar di berbagai lokasi. dengan adanya konfigurasi bintang ini, maka segala macam kegiatan yang ada di kantor cabang dapatlah di kontrol dan di koordinasikan dengan baik. di samping itu, dunia pendidikan juga banyak memanfaatkan jaringan bintang ini guna mengontrol kegiatan anak didik mereka.
  \item Ring Networks (jaringan Cincin)
  \     pada jaringan ini terdapat beberapa peralatan saling
        di hubungkan satu dengan lainnya dan pada akhirnya akan membentuk bagan seperti halnya sebuah cincin. jaringan cincin tidak memiliki suatu titik yang bertindak sebagai pusat ataupun pengantur lalu lintas data, semua simpul mempunyai tingkatan yang sama. data yang di kirim akan berjalan melewati beberapa simpul sehingga sampai pada simpul yang di tuju. dalam menyampaikan data, jaringan bisa bergerak dalam satu ataupun dua arah.

        walaupun demikian, data yang ada tetap bergerak satu arah dalam satu saat. pertama, pesan ada akan di sampaikan dari titik ke titik lainnya dalam satu arah. apabila di temui kegagalan, misalnya terdapat kerusakan pada peralatan yang ada, maka data yang akan di kirim dengan cara kedua, yaitu pesan kemudian di transmisi dalam arah yang berlawanan, dan pada akhirnya bisa berakhir pada tempat yang di tuju.
  \item tree Network (jaringan pohon)
        pada jaringan pohon, terdapat beberapa tingkatan simpul (node).pusat atau simpul yang lebih tinggi tingkatanya, dapat mengatur simpul lain yang lebih rendah tingkatanya.
\end{itemize}

topologi logik pada umumnya terbagi menjadi dua tipe,yaitu:
\begin{enumerate}
  \item Topologi Broadcast
        secara sederhana dapat di gambarkan yaitu suatu host yang mengirimkan data kepada seluruh host lain pada media jaringan.
  \item Topologi token passing
        mengatur pengiriman data pada host melalui media dengan menggunakan token yang secara teratur berputar pada seluruh host. host hanya dapat mengirimkan data hanya jika host tersebut memiliki token. dengan token ini, collision dapat di cegah.
\end{enumerate}

\subsection{Bentuk komunikasi data}
\begin{itemize}
  \item Simpleks line (komunikasi satu arah)
        Merupakan bentuk saluran komunikasi yang paling murah, di mana komunikasi jenis ini hanya bisa berlangsung satu arah, dengan demikian pengirim informasi tidak bisa bertindak ataupun berubah menjadi penerima informasi, demikian pula sebaliknya.

        walaupun murah, jenis ini simpleks line jarang dipergunakan untuk komunikasi data,kalapun terpaksa hanya dipergunakan untuk hubungan antara CPU dengen printer, di mana printer hanya akan bertindak sebagai penerima informasi dari CPU. dalam kehidupan sehari-hari, kita bisa melihat radio panggil (pager) yang menggunakan transmisi-line dengan bentuk simpleks
  \item half-Dupleks(dua Arah bergantian).
        hal-dupleks line menginjika transmisi data dilakukan dalam dua arah, tetepi tidak dalam waktu yang bersamaan. jika line yang ada sedang mengirim data, misalnya dari terminal ke CPU, maka line yang bersangkutan pada saat itu tidak bisa di gunakan untuk mengirim data kembali dari CPU keterminal.

        dalam kehidupan sehari-hari, kita bisa melihat radio-CB yang di gunakan oleh para satpam ataupun anggota kepolisian. Radio-CB yang mereka pergunakan, menggunakan bentuk saluran half-dupleks sehingga pada saat pembicaraan berlangsung, sang pembicaraan harus menekan tombol tertentu agar suara yang di kirimkan bisa di salurkan kepada penerima. apabila di rasa cukup, maka pembicara akan mengucapkan kata "ganti"
        sebagai tanda bahwa saluran tersebut bisa di gunakan oleh lawan berbicaranya.
  \item Full-Dupleks (dua arah penuh)
        di dalam komunikasi ini, penerima dan pengirim informasi bisa secara serentak melakukan kegiatan bersama-sama, ataupun saling bertukar posisi dari penerima menjadi pengirim berita dan sebaliknya. data dalam hal ini dapat di kirim dari dua arah pada saat yang bersamaan
\end{itemize}

\subsection{Media Transmisi Data}
\begin{itemize}
  \item kabel Twisted Pair/uritan.
        Kabel jenis ini merupakan kabel yang paling luas penggunaanya karena di pergunakan untuk jaringan telpon.
        kabel ini terbuat dari tembaga di mana beberapa pasang kabel di-untir dan di jadikan satu. Guna mempertinggi kualitas kabel, seringkali setiap pasang kabel akan saling di-untir sehingga disebut sebagai untir-an.
  \item kabel koaksial
        pada jenis ini,kabel utama yang terbuat dari tembaga akan di keliling oleh anyaman halus kabel tembaga lainnya dan di antaranya terdapat isolasi. dari sudut harga, kabel ini lebih mahal apabila dibandingkan dengan kabel untiran,tetepi kualitas yang di berikan juga lebih baik.
  \item Fiber Optic(serat optik)
        dewasa ini terdapat usaha untuk menggunakan cahaya sebagai media komunikasi. data yang ada akan di bawa oleh cahaya, dan untuk menyalurkan cahaya yang membawa data tersebut, di perlukan adanya suatu jenis kabel yang khusus, dan kabel inilah yang di sebut sebagai fiber optic cable ataupun serat fiber. fiber optic terdiri atas suatu gelas fiber yang sangat tipis dan dapat di pergunakan untuk menyalurkan data dalam jumlah dan kecepatan yang sangat tinggi.
  \item Gelombang Radio-AM
        Sinyal yang berbentuk analog, juga dapat di transmisikan melalui udara, seperti misalnya: gelombang radio. AM-Radio
        yang merupakan singkatan dari Amplitude Modulation, dapat menangkap sinyal pada frekwensi yang sama, dan dengan kekuatan dan amplitude yang di miliknya, dapatlah menggerakan informasi ke arah yang di tuju.
  \item Pemancar Radio-FM/Station Televisi.
        Pemancar radio-FM dan station televisi juga dapat di gunakan untuk menyalurkan gelombang analog. dalam hal ini, Station televisi ataupun pemancar radio-FM (frekwensi Modulation)
        akan mendiami gelombang antara 54 hingga 806 megahertz(milions of cycles per second)
  \item Radio komunikasi Gelombang Pendek.
        dalam hal ini, radio komunikasi gelombang pendek banyak di gunakan oleh kalangan tertentu, misalnya ORARI ataupun kepolisian, juga dapat di manfaatkan untuk membawa sinyal analog ketempat yang di tiju. Radio komunikasi gelombang pendek memliki frekwensi yang lebih tinggi jika di banding
        dengan frekwensi yang di miliki oleh pemancar radio-AM
  \item telephone Celluar
        Telpon celuler ataupun genggam, ataupun telfon mobil yang bekerja pada frekwensi 825 hingga 890 megahertz, juga dapat di maanfaatkan sebagai suatu media transmsi komunikasi data.
  \item Gelombang Mikro
        Komunikasi data melalui gelombang elektro magnet (udara)
        yang paling banyak di gunakan adalah dengan gelombang mikro atau microwave. cara ini bisa menjangkau jarak yang sangat jauh, sehingga banyak kalangan industri ataupun pribadi yang menggunakannya untuk memindahkan/menyalurkan suara, video ataupun data komunikasi
  \item Satelit
        Pengguna satelit dirancang untuk mengurangi biaya pada pengiriman jarak yang sangat jauh. apabila di gunakan gelombang mikro, maka di perlukan banyak sekali station pemancar bumi yang harus di bangun. di samping itu juga
        harus diingat adanya lautan yang memisahkan daratan satu
        dengan lainnya. dengan menggunakan satelit, maka permasalahan yang ada bisa di atasi. Satelit secara umum
        bekerja pada frekwensi antara dua hingga 40 gigahertz
        (billion of hertz)
  \end{itemize}

\subsection{Protokol}
1. pengertian dasar Protokol
    protokol adalah sebuah aturan yang mendefinisikan beberapa fungsi yang ada dalam sebuah jaringan komputer, misalnya mengirim pesan,data, informasi dan fungsi lain yang harus di penuhi oleh sisi pengirim (transmitter) dan sisi penerima (receiver) agar komunikasi berlangsung dengan benar. selain itu protokol juga berfungsi untuk memungkinkan dua atau lebih komputer dapat berkomunikasi dengan bahasa yang sama.

hal-hal yang harus di perhatikan dalam protokol adalah sebagai berikut:
\begin{itemize}
  \item Syntax
        merupakan format data dan cara pengkodean yang di gunakan untuk mengkodekan sinyal.
  \item semantix
        Di gunakan untuk mengetahui maksud dari informasi yang di kirim dan mengkoreksi kesalahan yang terjadi dari informasi tadi.
  \item Timing
        di gunakan untuk mengetahui kecepatan transmisi data.
\end{itemize}

\subsection{Fungsi protokol}
 2.fungsi-fungsi protokol secara detail dapat di jelaskan sebagai berikut:
 \begin{itemize}
   \item Fragmentasi dan Reassembly
         fungsi dari fragmentasi dan reassembly adalah membagi informasi yang di kirim menjadi beberapa paket data pada saat isi pengirim mengirimkan informasi tadi dan setelah di terima maka sisi penerima akan menggabungkan lagi menjadi paket berita yang lengkap.
   \item Encaptulation
         fungsi dan ecaptulation adalah melengkapi berita yang di kirimkan dengan address, kode-kode koreksi dan lain-lain.
   \item Connection Control
         fungsi dari connection control adalah membangun hubungan komunikasi dari transmitter dan receiver, di mana dalam membangun hubungan ini termasuk dalam hal pengiriman data dan mengakhiri hubungan.
   \item Flow control
         fungsi dari flow control adalah mengatur perjalanan data transmitter ke receiver
   \item Error Control
         dalam pengiriman data tak lepas dari kesalahan,baik itu dalam proses pengiriman maupun pada waktu data itu di terima.fungsi dari error control adalah mengontrol terjadinya kesalahan yang terjadi pada waktu data dikirimkan.
   \item Transmission Service
         fungsi dari transmission service adalah memberi pelayanan komunikasi data khususnya yang berkaitan dengan prioritas dan keamanan serta perlindungan data.
 \end{itemize}

\subsection{TCP/IP}
3.TCP/IP bukanlah sebuah protokol tunggal tetapi satu kesehatan protokol dan utility.setiap protokol dalam kesatuan ini memiliki aturan yang spesifik. protokol ini dikembangkan oleh ARPA(Advanced Research Project Agency) untuk departement pertahanan Amerika serikat pada tahun 1969.
ARPA menginginkan sebuah protokol yang memiliki karakter sebagai berikut:
\begin{itemize}
  \item Mampu menghubungkan berbagai jenis sistem operasi.
  \item Dapat diandalkan dan mampu mendukung komunikasi kecepatan tinggi
  \item Routable dan scalable untuk memenuhi jaringan yang kompleks dan luas.Sebuah alamat TCP/IP adalah nilai biner berukuran 32 Bit yang di berikan kesetiap host dalam sebuah jaringan. Nilai ini di gunakan untuk mengenali jaringan di mana host yang terhubung jadi satu pada sebuah internet work harus memiliki satu alamat unik TCP/IP.
\end{itemize}

Setiap alamat terbagi atas dua komponen:
\begin{itemize}
  \item Network ID
        ini adalah bagian dari alamat IP yang mewakili jaringan fisik dari host (nama jalan dari rumah). Setiap komputer dalam segmen jaringan tertentu akan memiliki ID jaringan yang sama.
  \item Node ID
        ini adalah bagian yang mewakili bagian individu dari alamat(nomor rumah). Bila komputer disegment jaringan memiliki alamat, maka jaringan tersebut perlu miliki siapakah suatu paket itu.
        seperti yang di sebutkan di atas tadi bahwa nilai IP adalah nilai biner 32 bit. Nilai tersebut terbagi menjadi empat bagian 8 bit yang di sebut oktet. contoh alamat IP:202.149.240.66 dengan mengunakan contoh di atas, katakanlah administrator mensetup jaringan dengan semua komputer memiliki bagian nilai yang sama 202.149.240.xxx.konidisi ini yang di sebut network ID. Nomor pada XXX adalah node ID-nya.
        setiap alamat TCP/IP jatuh pada satu kelas alamat. kelas mewakili sebuah grup alamat yang segera dapat di kenal komponen software sebagai bagian dari sebuah jaringan fisik.
        misalkan, ambil contoh alamat TCP/IP berikut dan nilai binernya. 10.149.240.660001010.10010101.1111000.10000010 dengan memperhatikan tiga nilai biner yang pertama, bisa di katakan bahwa alamat ini termasuk class A.
\end{itemize}
