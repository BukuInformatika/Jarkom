\section{Pengenalan Jaringan Komputer}
 Jaringan Komputer merupakan kumpulan dari beberapa PC(Personal Computer) atau peripheral yang saling terhubung melalui media transmisi(melalui kabel atau nirkabel) dan melakukan akses bersama  terhadap suatu resource.
 \par Secara lebih sederhana, jaringan komputer dapat diartikan sebagai sekumpulan komputer berserta mekanisme dan prosedurnya yang saling terhubung dan berkomunikasi.  Komunikasi yang dilakukan oleh komputer tersebut dapat berupa transfer berbagai data, instruksi, dan informasi dari satu komputer ke komputer yang lain \cite{irawan2012analisis}.

 resource(sumber daya) tersebut terdiri dari:
 \begin{enumerate}
   \item Hardware, seperti: Printer,mesin fax, store device.
   \item Software, seperti: game, pemprograman client server, multi user, mail server
   \item Stored, Seperti: frontend atau backend
   \item Internet,Seperti: dial atau wireless
 \end{enumerate}

Keuntungan Jaringan Komputer:
\begin{enumerate}
  \item Lebih hemat dalam biaya pengadaan dan pemeliharaan
  \item Memungkinkan management sumber daya lebih efisien
  \item Mempertahankan kualitas Informasi agar tatap handal
  \item Memungkinkan Kelompok kerja berkomunikasi lebih efisien
  \item Keamanan data lebih terjamin
\end{enumerate}

Type Jaringan Komputer
 Pada dasarnya seseorang menentukan type jaringan komputer karena beberapa alasan, diantaranya adalah:
 \begin{enumerate}
   \item Disesuaikan dengan kebutuhan kita dalam membuat jaringan komputer.
   \item Tergantung kepada jumlah pengguna yang akan melakukan sharing data.
   \item keamanan (securitas) dari masing-masing jaringan.
   \item Mempertimbangkan dalam biaya pengadaan dari jaringan komputer
   \item Sumber daya admin menentukan jaringan komputer.
   \item Bentuk dari organisasi yang terbentuk.
 \end{enumerate}

Server Based
pada type jaringan komputer server based di perlukan satu atau lebih komputer khusus yang di sebut server untuk mengatur lalu lintas data atau informasi dalam jaringan komputer.komputer-komputer selain server dinamakan client. server yaitu komputer yang menyediakan fasilitas bagi komputer-komputer lain, sedangkan client yaitu komputer-komputer yang menerima atau menggunakan fasilitas yang di sediakan oleh server \cite{yudianto2007jaringan}.

server dibedakan atas dua macam yaitu dedicated server(server bisa jadi client) dan undedicated server(server mutlak,tidak bisa jadi client).

macam-macam undedicated server:
\begin{itemize}
  \item DNS (Domain Name Service) yaitu server yang di gunakan untuk mengkonfersi penamaan IP address menjadi penanaman yang lebih familier (umum).
  \item DHCP (Dinamic Host Configurasi Protocol) yaitu server yang di gunakan untuk memberikan pengalaman IP address secara otomatis yang bersifat random. cara kerja random. Cara kerjanya pertama request (permintaan) kemudian dibroadcast.
  \item  FTP (file Transfer Protokol) yaitu server yang di gunakan untuk mengola jenis file/folder supaya data yang dinformasikan terpusat
  \item Mail Server merupakan jenis data dalam bentuk surat elektronik dibedakan menjadi dua yitu dalam bentuk text POP V3 (post office protocol) dan dalam bentuk web SMTP (simple Mail Transfer Protocol)
  \item Web server yaitu server yang di gunakan untuk mengelola data web yang bersifat dinamis.
  \item Database server yaitu server dalam bentuk file database.
\end{itemize}

ciri-ciri Server based
\begin{itemize}
  \item Operating System yang di gunakan berjenis network
  \item Perangkat yang di gunakan lebih dari 10 PC
  \item Terdapat komputer yang di jadikan sebagai pengontrol(server)\cite{wahyono2007building}
\end{itemize}

kelebihan Server based
\begin{itemize}
  \item terpusatnya penyedia resource
  \item Sharing data lebih efektif dan efesien
  \item System keamanan dan admistrasi jaringan lebih baik
\end{itemize}

Ciri-ciri Server Based
\begin{itemize}
  \item Operating System yang di gunakan berjenis network
  \item Perangkat yang di gunakan lebih dari 10 PC
  \item Terdapat Komputer yang di jadikan sebagai pengontrol(server)
\end{itemize}

Kelebihan Server Based
\begin{itemize}
  \item Terpusatnya penyedia resource
  \item Sharing data lebih efektif dan efesien
  \item System keamanan dan administrasi jaringan lebih baik
\end{itemize}

\subsection {Jenis-jenis Jaringan Komputer}
 Jenis jenis jaringan komputer dilihat berdasarkan ruang lingkup dan luas jangkuannya,di bedakan menjadi beberapa macam,yaitu:
\begin{itemize}
  \item Local Area Network(LAN)
   LAN adalah suatu system jaringan di mana setiao komputer atau perangkat keras dan perangkat lunak di gabungkan agar dapat saling berkomunikasi (terintegrasi) dalam area kerja tertentu dengan menggunakan data dan program yang sama,juga mempunyai kecepatan transfer data lebih cepat. Ruang Lingkup LAN anatr ruangan,gedung,kantor
\end{itemize}

\subsection {Topologi Jaringan Komputer}
    Topologi jaringan komputer adalah jaringan yang berhubungan dengan susuanan fisik semua jaringan komputer, baik server maupun client yang menentukan design,susunan,bentuk dari cara penempatan komputer(peripheral) kedalam jaringan-jaringan komputer.Topologi akan membentuk:
\begin{enumerate}
  \item Jenis alat yang di gunakan 
  \item kemampuan dari peralatan
  \item Pertembuhan dari jaringan komputer
  \item Bagaimana jaringan tersebut diatur
\end{enumerate} 
