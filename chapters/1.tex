\section{Perintah Navigasi}
Perintah navigasi direktori
ZIP ( Zone Information Protokol )

Protokol layer sesion pada AppleTalk yang memetakkan nomor jaringan menjadi Zona. ZIP digunakan oleh NBP untuk menentukkan jaringan mana yang berisikan node yang menjadi bagian dari suatu Zona,Wireless Access Protokol Bahasa yang digunakan untuk menulis halaman web yang menggunakan jauh lebih sedikit overhead, sehingga membuatnya lebih cocok untuk akses nirkabel ke internet.
Modul Praktikum Jarkom UIN Sunan Kalijaga YogyakartaTYPE JARINGAN KOMPUTERPada   dasarnya   seseorang   menentukan   type   jaringan   komputer   karena beberapa alasan, diantaranya adalah:
\begin{enumerate}
\item Disesuaikan dengan kebutuhan kita dalam membuat jaringan komputer.

\item Tergantung kepada jumlah pengguna yang akan melakukan sharing data.

\item Keamanan (securitas) dari masing-masing jaringan.

\item Mempertimbangkan dalam biaya pengadaan dari jaringan komputer.

\item Sumber daya admin menentukan jaringan komputer.f. Bentuk dari organisasi yang terbentuk

\end{enumerate}
Macam-macam type jaringan komputer adalah sebagai berikut :
Peer to Peer
(gambar peer to peer)
Pada type jaringan Peer to Peer, antara PC A dan PC B berkedudukan sama dan system yang digunakan switch, yang satu menjadi sumber dan yang lainnya menjadi akses. Dapat dikoneksikan melalui Direct Cable Connection (DCC) atau Norton Commander (NC).
DCC   yang  menjadi  sumber   dinamakan  Host  dan   yang   akses  dinamakan Guest. Sedangkan pada NC, sumber (Master) dan akses (Slave).Ciri – ciri Peer to Peer•komputer yang digunakan maximal 2 unit.•Operating System yang digunakan jenis desktop yang bersifat Transmitter dan Receiver.•Utility bisa menggunakan DCC atau NC •Bersifat “Hemaphodit” yaitu dapat ditukar antara Master dan Slave.

Ciri – ciri Peer to Peer
\begin{itemize}
\item komputer yang di gunakan maximal 2 unit
\item Operating System yang di gunakan jenis yang bersifat \emph{Transmitter} dan \emph{Receiver}
\item Utility bisa menggunakan DCC atau NC
\item Bersifat “Hemaphodit” yaitu dapat ditukar antara Master dan Slave
    
