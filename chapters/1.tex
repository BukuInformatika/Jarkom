\section{Pengenalan Jaringan Komputer}
 Jaringan Komputer merupakan kumpulan dari beberapa PC(Personal Computer) atau peripheral yang saling terhubung melalui media transmisi(melalui kabel atau nirkabel) dan melakukan akses bersama  terhadap suatu resource.
 \par Secara lebih sederhana, jaringan komputer dapat diartikan sebagai sekumpulan komputer berserta mekanisme dan prosedurnya yang saling terhubung dan berkomunikasi.  Komunikasi yang dilakukan oleh komputer tersebut dapat berupa transfer berbagai data, instruksi, dan informasi dari satu komputer ke komputer yang lain \cite{irawan2012analisis}.

 resource(sumber daya) tersebut terdiri dari:
 \begin{enumerate}
   \item Hardware, seperti: Printer,mesin fax, store device.
   \item Software, seperti: game, pemprograman client server, multi user, mail server
   \item Stored, Seperti: frontend atau backend
   \item Internet,Seperti: dial atau wireless
 \end{enumerate}

Keuntungan Jaringan Komputer:
\begin{enumerate}
  \item Lebih hemat dalam biaya pengadaan dan pemeliharaan
  \item Memungkinkan management sumber daya lebih efisien
  \item Mempertahankan kualitas Informasi agar tatap handal
  \item Memungkinkan Kelompok kerja berkomunikasi lebih efisien
  \item Keamanan data lebih terjamin
\end{enumerate}

Type Jaringan Komputer
 Pada dasarnya seseorang menentukan type jaringan komputer karena beberapa alasan, diantaranya adalah:
 \begin{enumerate}
   \item Disesuaikan dengan kebutuhan kita dalam membuat jaringan komputer.
   \item Tergantung kepada jumlah pengguna yang akan melakukan sharing data.
   \item keamanan (securitas) dari masing-masing jaringan.
   \item Mempertimbangkan dalam biaya pengadaan dari jaringan komputer
   \item Sumber daya admin menentukan jaringan komputer.
   \item Bentuk dari organisasi yang terbentuk.
 \end{enumerate}

Server Based
pada type jaringan komputer server based di perlukan satu atau lebih komputer khusus yang di sebut server untuk mengatur lalu lintas data atau informasi dalam jaringan komputer.komputer-komputer selain server dinamakan client. server yaitu komputer yang menyediakan fasilitas bagi komputer-komputer lain, sedangkan client yaitu komputer-komputer yang menerima atau menggunakan fasilitas yang di sediakan oleh server \cite{yudianto2007jaringan}.

server dibedakan atas dua macam yaitu dedicated server(server bisa jadi client) dan undedicated server(server mutlak,tidak bisa jadi client).

macam-macam undedicated server:
\begin{itemize}
  \item DNS (Domain Name Service) yaitu server yang di gunakan untuk mengkonfersi penamaan IP address menjadi penanaman yang lebih familier (umum).
  \item DHCP (Dinamic Host Configurasi Protocol) yaitu server yang di gunakan untuk memberikan pengalaman IP address secara otomatis yang bersifat random. cara kerja random. Cara kerjanya pertama request (permintaan) kemudian dibroadcast.
  \item  FTP (file Transfer Protokol) yaitu server yang di gunakan untuk mengola jenis file/folder supaya data yang dinformasikan terpusat
  \item Mail Server merupakan jenis data dalam bentuk surat elektronik dibedakan menjadi dua yitu dalam bentuk text POP V3 (post office protocol) dan dalam bentuk web SMTP (simple Mail Transfer Protocol)
  \item Web server yaitu server yang di gunakan untuk mengelola data web yang bersifat dinamis.
  \item Database server yaitu server dalam bentuk file database.
\end{itemize}

ciri-ciri Server based
\begin{itemize}
  \item Operating System yang di gunakan berjenis network
  \item Perangkat yang di gunakan lebih dari 10 PC
  \item Terdapat komputer yang di jadikan sebagai pengontrol(server)\cite{wahyono2007building}
\end{itemize}

kelebihan Server based
\begin{itemize}
  \item terpusatnya penyedia resource
  \item Sharing data lebih efektif dan efesien
  \item System keamanan dan admistrasi jaringan lebih baik
\end{itemize}

Ciri-ciri Server Based
\begin{itemize}
  \item Operating System yang di gunakan berjenis network
  \item Perangkat yang di gunakan lebih dari 10 PC
  \item Terdapat Komputer yang di jadikan sebagai pengontrol(server)
\end{itemize}

Kelebihan Server Based
\begin{itemize}
  \item Terpusatnya penyedia resource
  \item Sharing data lebih efektif dan efesien
  \item System keamanan dan administrasi jaringan lebih baik
\end{itemize}

\subsection {Jenis-jenis Jaringan Komputer}
 Jenis jenis jaringan komputer dilihat berdasarkan ruang lingkup dan luas jangkuannya,di bedakan menjadi beberapa macam,yaitu:
\begin{itemize}
  \item Local Area Network(LAN)
   LAN adalah suatu system jaringan di mana setiao komputer atau perangkat keras dan perangkat lunak di gabungkan agar dapat saling berkomunikasi (terintegrasi) dalam area kerja tertentu dengan menggunakan data dan program yang sama,juga mempunyai kecepatan transfer data lebih cepat. Ruang Lingkup LAN anatr ruangan,gedung,kantor
\end{itemize}

\subsection {Topologi Jaringan Komputer}
    Topologi jaringan komputer adalah jaringan yang berhubungan dengan susuanan fisik semua jaringan komputer, baik server maupun client yang menentukan design,susunan,bentuk dari cara penempatan komputer(peripheral) kedalam jaringan-jaringan komputer.Topologi akan membentuk:
\begin{enumerate}
  \item Jenis alat yang di gunakan
  \item kemampuan dari peralatan
  \item Pertembuhan dari jaringan komputer
  \item Bagaimana jaringan tersebut diatur
\end{enumerate}

jenis alat-alat yang di gunakan,syaratnya:
\begin{itemize}
  \item Minimal 2 PC
  \item Adanya Operating System
  \item Adanya Network Interface Card(NIC)
  \item Driver NIC
  \item Media Transmisi
  \item Konsetrator(penghubung
  \item Access Point(tanpa kabel)
  \item Hub
  \item switch
  \item Repeater(Penguat signal)
  \item Router (Pembeda IP Address)
  \item Gatway(Perbedaan Arsitektur)
  \item Bridge (penghubung perbedaan topologi)
  \item Modern (modulasi de Modelator)
\end{itemize}

Topologi jaringan dibagi menjadi dua macam yaitu:
\begin{enumerate}
  \item Fisika
        Topologi ini menjelaskan tentang bentuk dari jaringan komputer yang dapat dilihat secara fisik/nyara.
\end{enumerate}

\begin{enumerate}
  \item Topologi Bus
        Masing-masing server dan workstastion di hubungkan pada sebuah kabel yang di sebut trunk atau backbone, kabel untuk menghubungkan jaringan ini biasanya menggunakan kabel Coaxial (kabel BNC). setiap server dan workstation yang di sambungkan pada bus menggunakan konektor T.
\end{enumerate}
 pada kedua ujung dari kabel harus diberi terminator berupa resistor yang memiliki resistansi khusus sebesar 50 Ohm yang berwujud sebuah konektor.Apabila resistansi kabel dibawah maupun di atas 50 Ohm,maka server tidak akan bisa bekerja secara maksimal dalam melayani jaringan,sehingga akses user dan client menjadi menurun.
 kelebihan jaringan topologi bus:
 \begin{itemize}
   \item Penggunaan kabel yang sedikit sehingga terlihat sederhana.
   \item Pengembangan jaringan mudah.
 \end{itemize}
 Kekurangan jaringan topologi
 \begin{itemize}
   \item Membutuhkan repeater untuk jarak jaringan yang terlalu jauh.
   \item jaringan akan terganggu apabila salah satu komputer mengalami kerusakan.
   \item Deteksi kesalahan sangat kecil sehingga apabila terjadi gangguan maka sulit sekali mencari kesalahan tersebut.
   \item Terjadi antrian data
 \end{itemize}

\begin{enumerate}
  \item Topologi Star
        Pada topologi in, setiap komputer(node) dalam jaringan terhubung ke sebuah pusat jaringan,yang biasa berupa hub,switch, dan juga berupa komputer. setiap workstasion dihubungkan ke server menggunakan suatu konsentrator. masing-masing workstastion tidak saling berhubungan.jadi setiap user yang terhubung ke server tidak akan dapat berinteraksi dan melakukan apa-apa sebelum server kita di hidupkan.apabila komputer server mati maka semua koneksi jaringan akan terputus.
\end{enumerate}
kelebihan jaringan topologi star:
\begin{itemize}
  \item Mudah dalam medeteksi kesalahan jaringan karena control jaringan terpusat.
  \item fleksibe dalam hal pemasangan jaringan baru tanpa mempengaruhi jaringan yang lain.
  \item apabila salah satu kabel koneksi user terpusat maka hanya user yang bersangkutan saja yang tidak akan berfungsi dan tidak mempengaruhi user yang lain.
\end{itemize}
kekurangan jaringan topologi star:
\begin{itemize}
  \item Boros dalam pemakaian kabel jika kita hubungkan dengan jaringan yang lebih besar dan luas.
  \item Control hanya terpusat pada hub/switch sehingga operasionalnya perlu ditangani secara khusus.
\end{itemize}

\begin{enumerate}
  \item topologi cincin atau yang sering disebut dengan ring topologi adalah topologi jaringan di mana setiap komputer yang terhubung membuat lingkaran. dengan artian setiap komputer yang terhubung kedalam satu jaringan saling terkoneksi ke dua komputer lainnya sehingga membentuk satu jaringan yang sama dengan bentuk cincin.
      Pada setiap komputer akan dihubungkan dan di jadikan repeater(penguat signal). komputer yang diberi frame berhak mengirim data dan komputer yang lain menjadi repeater. pada topologi ring terdapat token frame yang saling berkesinambungan dan pada prinsipnya menggunakan CSMA/CD(Carrier Sense Multyple Access/Collection Detection).
\end{enumerate}
Kelebihan jaringan topologi ring adalah:
\begin{itemize}
  \item Hemat kabel
  \item Dapat mengisolasi kesalahan dari suatu workstation kekurangan jaringan topolgi ring
  \item Sangat peka terhadap kesalahan jaringan walaupun sekecil apapun.
  \item Sukar untuk mengembangkan jaringan,sehingga jaringan tersebut tampak menjadi kaku.
  \item Biaya pemasangan Lebih besar.
\end{itemize}

\begin{enumerate}
  \item Logik
        sedanhkan topologi ini menjelaskan tentang bagaimana signal akan melewati komputer didalam jaringan. Arsitektur ini terus di kembangkan sampai saat ini.
\end{enumerate}

\begin{enumerate}
  \item  Token Ring
         Token ring memanfaatkan topologi ring. sebuah token bebas mengalir dalam jaringan. Apabila suatu node ingin mengirim paket data, maka paket data yang akan dikirim ditempel pada token, token kemudian akan membawa paket data tersebut pada tujuannya. pada waktu token terisi data, node lain tidak dapat menggunakan token tersebut sampai token menyelesaikan tugas mengirimkan paket data. apabila paket data telah di sampaikan pada tujuan,node pengguna tadi melepaskan token untuk dipakai oleh node lain. cara kerja dinamakan token passing scheme.
\end{enumerate}
ciri-ciri token ring:
\begin{itemize}
  \item Kecepatannya 1 Mbps, 4 Mbps hingga 16 Mbps.
  \item untuk menghubungkan station membutuhkan multistation Access Unit(MAU)
\end{itemize}

\begin{enumerate}
  \item Arsitektur ArcNet(Attached ResourceComputer Network)
        Didesain untuk system komputer Datapoint dan dikembangkan oleh Datapoint Corporation. Saat pertama didesain Arcnet menggunakan ukuran frame kecil 508 byte. Arcnet didesain agar handal dan tahan terhadap kerusakan pada kabel dan station.
\end{enumerate}
Ciri-ciri ArcNet adalah:
\begin{itemize}
  \item Topologi fisik yang di gunakan biasanya topologi Bus atau Star
  \item Prinsip kerjanya menggunakan token passing scheme atau broadcast
  \item Implementasinya menggunakan kabel coaxial RG-62
  \item Kecepatan mulai dari 2.5 Mbps hingga 20 Mbps.
\end{itemize}

\begin{enumerate}
  \item Merupakan implementasi metode CSMA/CD yang dikembangkan
        tahun 1960 pada proyek wire.sejak tahun 1978 IEEE (Institute Of Electrical and Electronics Engineers) telah melakukan
        standarisasi system Ethernet. Kecepatan Transmisi data saat ini antara 10 sampai 100 Mbps.
\end{enumerate}

\begin{enumerate}
  \item FDDI(Fiber Distribusi Data Interface)
        Merupakan suatu protocol jaringan yang menghubungkan antara
        dua atau beberapa jaringan yang jaraknya berdekatan ataupun berjauhan adapun metode yang di gunakan dalam FDDI adalah metode token ring.
\end{enumerate}

ciri-ciri FDDI adalah:
\begin{itemize}
  \item  Implementasinya menggunakan kabel fiber optic
  \item  Memiliki kecepatan 100 Mbps
  \item  Tidak compatibel dengan Ethernet tapi Ethernet dapat
         dienkapsulasi dalam paket FDDI
  \item  Bekerja berdasarkan dua ring concentris
  \item  Apabila salah satu ring atau node putus maka ring yang lain
         dapat berfungsi sebagai back up.
\end{itemize}

\begin{enumerate}
  \item ATM(Asynchronous Transfer MOde)
        Merupakan teknologi jaringan berkecepatan tinggi yang mampu
        mengirim data,suara dan video secara real time.ATM juga biasa di sebut Cell Relay. ATM merupakan interface transfer paket yang effesien. ATM menggunakan paket-paket dengan ukuran tertentu yang di sebut dengan cell. karena menggunakan ukuran tertentu ini, ATM menghasilkan skema yang efisien bagi pentransmisian pada jaringan berkecepatan tinggi. ATM menyediakan layanan real time dan non real time.
\end{enumerate}
