\section{Perintah Navigasi}
Perintah navigasi direktori
ZIP ( Zone Information Protokol )

Protokol layer sesion pada AppleTalk yang memetakkan nomor jaringan menjadi Zona. ZIP digunakan oleh NBP untuk menentukkan jaringan mana yang berisikan node yang menjadi bagian dari suatu Zona,Wireless Access Protokol Bahasa yang digunakan untuk menulis halaman web yang menggunakan jauh lebih sedikit overhead, sehingga membuatnya lebih cocok untuk akses nirkabel ke internet.
Modul Praktikum Jarkom UIN Sunan Kalijaga YogyakartaTYPE JARINGAN KOMPUTERPada   dasarnya   seseorang   menentukan   type   jaringan   komputer   karena beberapa alasan, diantaranya adalah:
\begin{enumerate}
\item Disesuaikan dengan kebutuhan kita dalam membuat jaringan komputer.

\item Tergantung kepada jumlah pengguna yang akan melakukan sharing data.

\item Keamanan (securitas) dari masing-masing jaringan.

\item Mempertimbangkan dalam biaya pengadaan dari jaringan komputer.

\item Sumber daya admin menentukan jaringan komputer.f. Bentuk dari organisasi yang terbentuk

\end{enumerate}
Macam-macam type jaringan komputer adalah sebagai berikut :
Peer to Peer
(gambar peer to peer)
Pada type jaringan Peer to Peer, antara PC A dan PC B berkedudukan sama dan system yang digunakan switch, yang satu menjadi sumber dan yang lainnya menjadi akses. Dapat dikoneksikan melalui Direct Cable Connection (DCC) atau Norton Commander (NC).
DCC   yang  menjadi  sumber   dinamakan  Host  dan   yang   akses  dinamakan Guest. Sedangkan pada NC, sumber (Master) dan akses (Slave).Ciri – ciri Peer to Peer•komputer yang digunakan maximal 2 unit.•Operating System yang digunakan jenis desktop yang bersifat Transmitter dan Receiver.•Utility bisa menggunakan DCC atau NC •Bersifat “Hemaphodit” yaitu dapat ditukar antara Master dan Slave.

Ciri – ciri Peer to Peer
\begin{itemize}
\item komputer yang di gunakan maximal 2 unit

\item Operating System yang di gunakan jenis yang bersifat \emph{Transmitter} dan \emph{Receiver}

\item Utility bisa menggunakan DCC atau NC

\item Bersifat “Hemaphodit” yaitu dapat ditukar antara Master dan Slave

\end{itemize}

Keuntungan Menggunakan Peer to Peer
\begin{itemize}
\item Kelangsungan kerja tidak tergantung pada satu server,
      karena jika salah satu komputer mati atau rusak maka jaringan secara keseluruhan tidak akan mengalami gangguan.
\item Lebih mudah dalam melakukan konfigurasi.

\item Lebih mudah dalam melakukan konfigurasi.

\item Biaya operasional lebih murah.

\item Komputer dalam jaringan dapat saling berbagi fasilitas yang dimiliki seperti harddisk, drive, fax/modem, printer dll.

\end{itemize}

kelemahan Menggunakan Peer to Peer.
\begin{itemize}
\item Troubleshooting, jaringan lebih sulit karena pada jaringan Peer to Peer setiap komputer memungkinkan untuk terlibat dalam komunikasi yang ada. Pada jaringan Client-Server, komunikasinya hanya antara server dengan workstation.

\item System bergilir ( swap atau change ).

\item System keamanan jaringan ditentukan oleh masing-masing user dengan mengatur keamanan masing-masing fasilitas yang dimiliki.

\end{itemize}

Work Group
     Pada Type jaringan komputer workgroup setiap personal computer yang terkoneksi melalui workgroup dinamakan Workstasion.

  Ciri-ciri Workgroup
\begin{itemize}
\item Perangkat yang di gunakan lebih dari 2 PC dan di dalam satu workgroup maximal 10 PC.

\item Operating System desktop

\end{itemize}

kelebihan WorkGroup
\begin{item}

\item  Mudah dikonfigurasi

\item  Control resourcenya masing-masing

\end{itemize}

Kekurangan WorkGroup
\begin{itemize}

\item Data tidak terpusat

\item Performa akan menurun jika terlalu banyak akses

\end{itemize}

Server based

pada type jaringan komputer server based di perlukan satu atau lebih komputer khusus yang di sebut  server untuk mengatur lalu lintas data
atau informasi dalam jaringan komputer. Komputer-komputer selain server dinamakan client. Server yaitu komputer yang menyediakan fasilitas  bagi komputerkomputer lain, sedangkan client yaitu komputer-komputer yang menerima atau menggunakan fasilitas yang disediakan oleh server. menggunakan fasilitas yang disediakan oleh server.
server di bedakaan atas dua macam yaitu dedicated server (server bisa jadi client) dan undedicated server(server mutlak,tidak bisa jadi client).

macam-macam undedicated server:
\begin{itemize}

\item DNS (Domain Name Service) yaitu server yang di gunakan untuk
      mengkonfersi penamaan IP Address menjadi  penamaan yang lebih
      familier (umum).

\item  DHCP ( Dinamic Host Configurasi Protocol )  yaitu server yang
       digunakan untuk memberikan pengalamatan IP Address  secara otomatis yang bersifat random. Cara kerjanya pertama request
       (permintaan)kemudian dibroadcast.

\item   FTP ( File Transfer Protocol ) yaitu server yang  digunakan
        untuk mengelola jenis file/folder supaya data yang  diinformasikan terpusat.

\item   Mail Server merupakan jenis data dalam bentuk surat
        elektronik. Dibedakan menjadi dua yaitu  dalam bentuk text POP V3 (Post Office Protocol) dan dalam bentuk web SMTP (Simple Mail Transfer Protocol).
         
\item   Web Server yaitu server yang digunakan untuk mengelola data 
        yang bersifat dinamis.
        
\item   Database Server yaitu server dalam bentuk file database.
 
\end{itemize}

ciri-ciri Server Based
\begin{itemize}

\item Operating System yang digunakan berjenis network 

\item Perangkat yang digunakan lebih dari 10 PC 

\item Terdapat komputer yang dijadikan sebagai pengontrol (server)

\end{itemize}

kelebihan Server Based
\begin{itemize}

\item Terpusatnya penyedia resource 

\item Sharing data lebih efektif dan efisien 

\item System keamanan dan administrasi jaringan lebih baik

\item System backup data lebih baik 

\end{itemize}

Kekurangan Server Based
\begin{itemize}

\item Biaya operasional lebih mahal

\item Dibutuhkan satu komputer khusus yang  berkemampuan lebih untuk
      dijadikan server dan tenaga admin yang baik
      
\item Sangat ketergantungan pada server, karena jika server
      mengalami gangguan atau masalah maka secara keseluruhan 
       jaringan akan terganggu.
       
\end{itemize}








