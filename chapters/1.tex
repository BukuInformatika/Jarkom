\section{Pengenalan Jaringan Komputer}
 Jaringan Komputer merupakan kumpulan dari beberapa PC(Personal Computer) atau peripheral yang saling terhubung melalui media transmisi(melalui kabel atau nirkabel) dan melakukan akses bersama  terhadap suatu resource.
 \par Secara lebih sederhana, jaringan komputer dapat diartikan sebagai sekumpulan komputer berserta mekanisme dan prosedurnya yang saling terhubung dan berkomunikasi.  Komunikasi yang dilakukan oleh komputer tersebut dapat berupa transfer berbagai data, instruksi, dan informasi dari satu komputer ke komputer yang lain.

 resource(sumber daya) tersebut terdiri dari:
 \begin{enumerate}
   \item Hardware, seperti: Printer,mesin fax, store device.
   \item Software, seperti: game, pemprograman client server, multi user, mail server
   \item Stored, Seperti: frontend atau backend
   \item Internet,Seperti: dial atau wireless
 \end{enumerate}

Keuntungan Jaringan Komputer:
\begin{enumerate}
  \item Lebih hemat dalam biaya pengadaan dan pemeliharaan
  \item Memungkinkan management sumber daya lebih efisien
  \item Mempertahankan kualitas Informasi agar tatap handal
  \item Memungkinkan Kelompok kerja berkomunikasi lebih efisien
  \item Keamanan data lebih terjamin
\end{enumerate}

Type Jaringan Komputer
 Pada dasarnya seseorang menentukan type jaringan komputer karena beberapa alasan, diantaranya adalah:
 \begin{enumerate}
   \item Disesuaikan dengan kebutuhan kita dalam membuat jaringan komputer.
   \item Tergantung kepada jumlah pengguna yang akan melakukan sharing data.
   \item keamanan (securitas) dari masing-masing jaringan.
   \item Mempertimbangkan dalam biaya pengadaan dari jaringan komputer
   \item Sumber daya admin menentukan jaringan komputer.
   \item Bentuk dari organisasi yang terbentuk.
 \end{enumerate} 